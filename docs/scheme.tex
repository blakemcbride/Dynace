\parindent = 0in

\footline = {\tenrm \hfil Page \folio \hfil 3/30/99 \hfil}


\font\sfont = cmb10 at 12pt
\font\bfont = cmb10 at 17pt
\font\nfont = cmtt12
\font\ifont = cmti10 at 12pt
\font\tfont = cmr12
\tfont

\def\var#1{{\ifont #1\/}}

\def\section#1{\bigskip\centerline{\sfont\underbar{#1}}}
\def\pb{\#$\backslash$}
\def\bdisplay{\bgroup\leftskip = .5in\bigskip\obeywhitespace\nfont}
\def\edisplay{\bigskip\egroup}
\def\op{$\{$}
\def\cp{$\}$}

\def\split#1~#2~{%
\hbox{\hsize = 3in\hskip .5in\nfont%
\vtop{%
#1
}\hskip .5in\vtop{%
#2
}}}

\def\c{%
\hbox\bgroup\hsize = 3in\hskip .5in\nfont%
\vtop\bgroup\obeywhitespace%
}
\def\scheme{%
\egroup\hskip .5in\vtop\bgroup\obeywhitespace%
}
\def\endcode{%
\egroup\egroup%
}


\centerline{\bfont \underbar{Scheme Language Elements Compared to C}}

\section{Comments}

\split /*   comments  */ ~ ;  comments ~

\section{Data Types}

\split 2 ~ 2 ~
\split 3.141 ~ 3.141 ~
\split "Hello World" ~ "Hello World" ~
\split 'x' ~ \pb{x} ~
\split '$\backslash$n' ~ \pb{newline} ~
\split ' ' ~ \pb{space} ~

\section{Simple Expressions}

\split 3 + 5 ~ (+ 3 5) ~
\split 3 + 5 + 9 ~ (+ 3 5 9) ~
\split 10 - 2 ~ (- 10 2) ~
\split 10 * 2 ~ (* 10 2) ~
\split 22 / 7 ~ (/ 22 7) ~
\split fun(); ~ (fun) ~
\split fun(3, x); ~ (fun 3 x) ~

\section{Math Operators}

\split a + b + c ~ (+ a b c) ~
\split a - b ~ (- a b) ~
\split a * b * c ~ (* a b c) ~
\split a / b ~ (/ a b) ~
\split a \% b ~ (modulo a b) ~
\split x > y ~ (> x y) ~
\split x < y ~ (< x y) ~
\split x == y ~ (= x y) ~
\split x <= y ~ (<= x y) ~
\split x >= y ~ (>= x y) ~
\split x != y ~ (<> x y) ~
\split x == 0 ~ (zero? x) ~
\split x > 0 ~ (positive?\ x) ~
\split x < 0 ~ (negative?\ x) ~
\split x > 0 ? x : -x ~ (abs x) ~
\split \  ~ (max a b c ...) ~
\split \  ~ (min a b c ...) ~
\split \  ~ (floor x) ~
\split \  ~ (ceiling x) ~
\split \  ~ (round x) ~
\split \  ~ (truncate x) ~

\section{Logical Operators}

\split x \&\& y \&\& z ~ (and x y z) ~
\split x || y || z ~ (or x y z) ~
\split !x ~ (not x) ~

\vfil\break

\section{String Operators}

\split strcat(a, b) ~ (set!\ a (string-append a b c d)) ~
\split strlen(x) ~ (string-length x) ~
\split \  ~ (substring "abcdef" 0 4) -> "abcd" ~
\split !strcmp(x, y) ~ (string=?\ x y) ~
\split !strcmpi(x, y) ~ (string-ci=?\ x y) ~

\section{Global Declarations}

\split int  x = 4; ~ (define x 4) ~
\split char x[] = "string"; ~ (define x "string") ~

\section{Assignments}

\split x = 5; ~ (set!\ x 5) ~
\split x = b * 4; ~ (set!\ x (* b 4)) ~
\split x = b * (4 + g); ~ (set!\ x (* b (+ 4 g))) ~
\split x = fun(g); ~ (set!\ x (fun g)) ~
\split x = fun(g + 5); ~ (set!\ x (fun (+ g 5))) ~

\section{Function Definitions}

\bdisplay
(define  (<function>  <arguments>)
    <expressions>...)
\edisplay

The last expression evaluated is the value returned when the function
is executed.

\bigskip

\c
int  fun(int a, int b)
\op
     return a + b;
\cp
\scheme
(define (fun a b)
     (+ a b))
\endcode

\c
int  fun(int a, int b)
\op
    fun1();
    fun2();
    return a + b;
\cp
\scheme
(define (fun a b)
    (fun1)
    (fun2)
    (+ a b))
\endcode

\vfil\break
 
\section{Conditionals}

\split x == y ? 5 : 6; ~ (if (= x y) 5 6) ~
\c
x == y ? fun1() : fun2();
\scheme
(if (= x y)
    (fun1)
    (fun2))
\endcode
\c
if (x == y)
    fun1();
\scheme
(if (= x y)
    (fun1))
\endcode
\c
if (x == y)
    fun1();
else
    fun2();
\scheme
(if (= x y)
    (fun1)
    (fun2))
\endcode
\c
if (x == y) \op
    fun1();
    fun2();
\cp
\scheme
(if (= x y)
    (begin
        (fun1)
        (fun2)))
\endcode
\c
if (x == y) \op
    fun1();
    fun2();
\cp else \op
    fun3();
    fun4();
\cp
\scheme
(if (= x y)
    (begin
        (fun1)
        (fun2))
    (begin
        (fun3)
        (fun4)))
\endcode

\section{Multiple Conditional Statements}

\bdisplay
(cond 
        (condition1  consequent1)
        (condition2  consequent2)
             .            .
        (conditionn  consequentn)
        (else  alternative))
\edisplay

Each consequence can be any number of statements.  The return value of
\var{cond} is the last statement executed.



\section{Iteration}

\c
while (x == y) \op
    fun1();
    fun2();
\cp
\scheme
(while (= x y)
    (fun1)
    (fun2))
\endcode
\c
for (x=0 ; x != 10 ; x++) \op
    fun1();
    fun2();
\cp
\scheme
(do-times 10
    (fun1)
    (fun2))
\endcode

\vfil\break

\section{Local Variables}

\bdisplay
(let*  (<local-variable-list>)
       <expressions>...)

<local-variable-list> = (<variable>  <initial-value>)...
\edisplay

\var{let*} will create the local variables with the specified initial values
and then execute the expressions.  The return value of \var{let*} is the
last expression executed.

\bigskip

\c
int  fun(int a)
\op
    int b = 4;
    return b + a;
\cp
\scheme
(define (fun a)
    (let* ((b 4))
        (+ b a)))
\endcode
\c
int  fun(int a)
\op
    int b=4, d=5;
    return a + b + d;
\cp
\scheme
(define (fun a)
    (let* ((b 4)
           (d 5))
         (+ a b d)))
\endcode
\c
int  fun(int a)
\op
    int b=4, d=fun2();
    return a + b + d;
\cp
\scheme
(define (fun a)
    (let* ((b 4)
           (d (fun2)))
         (+ a b d)))
\endcode

\section{Data Type Tests}

\split \ ~	(number?\ x) ~
\split \ ~	(integer?\ x) ~
\split \ ~	(real?\ x) ~
\split \ ~	(string?\ x) ~
\split \ ~	(procedure?\ fun) ~


\bye
