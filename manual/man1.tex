@c -*-texinfo-*-

@c  Copyright (c) 1996 Blake McBride
@c  All rights reserved.
@c
@c  Redistribution and use in source and binary forms, with or without
@c  modification, are permitted provided that the following conditions are
@c  met:
@c
@c  1. Redistributions of source code must retain the above copyright
@c  notice, this list of conditions and the following disclaimer.
@c
@c  2. Redistributions in binary form must reproduce the above copyright
@c  notice, this list of conditions and the following disclaimer in the
@c  documentation and/or other materials provided with the distribution.
@c
@c  THIS SOFTWARE IS PROVIDED BY THE COPYRIGHT HOLDERS AND CONTRIBUTORS
@c  "AS IS" AND ANY EXPRESS OR IMPLIED WARRANTIES, INCLUDING, BUT NOT
@c  LIMITED TO, THE IMPLIED WARRANTIES OF MERCHANTABILITY AND FITNESS FOR
@c  A PARTICULAR PURPOSE ARE DISCLAIMED. IN NO EVENT SHALL THE COPYRIGHT
@c  HOLDER OR CONTRIBUTORS BE LIABLE FOR ANY DIRECT, INDIRECT, INCIDENTAL,
@c  SPECIAL, EXEMPLARY, OR CONSEQUENTIAL DAMAGES (INCLUDING, BUT NOT
@c  LIMITED TO, PROCUREMENT OF SUBSTITUTE GOODS OR SERVICES; LOSS OF USE,
@c  DATA, OR PROFITS; OR BUSINESS INTERRUPTION) HOWEVER CAUSED AND ON ANY
@c  THEORY OF LIABILITY, WHETHER IN CONTRACT, STRICT LIABILITY, OR TORT
@c  (INCLUDING NEGLIGENCE OR OTHERWISE) ARISING IN ANY WAY OUT OF THE USE
@c  OF THIS SOFTWARE, EVEN IF ADVISED OF THE POSSIBILITY OF SUCH DAMAGE.


@chapter Introduction
Dynace (pronounced @emph{d@=i-ne-s@=e}) stands for a ``DYNAmic C
language Extension''.  It is an object oriented extension to the C or
C++ languages.

Dynace is written in the C language, designed to be as portable as
possible and achieve its goals without changing or adding much to the
standard C syntax.

Dynace is a preprocessor, include files and a library which extends the
C or C++ languages with advanced object oriented capabilities, automatic
garbage collection and multiple threads.  Dynace is designed to solve
many of the problems associated with C++ while being easier to learn and
containing more flexible object oriented facilities.  Dynace is able to
add facilities previously only available in languages such as Smalltalk
and CLOS without all the overhead normally associated with those
environments.

This manual is designed for a programmer who is proficient in the C language,
but has little or no knowledge of object oriented concepts.  We will start
with some object oriented concepts, and then discuss how these concepts
are realized with Dynace.  We will then continue with a detailed description of
Dynace usage.  Finally, we will discuss the various classes which are built
on top of the Dynace kernel.


@section Features
The following table lists some of the key features of Dynace:

@itemize @bullet

@item Written in C and designed to maximize portability (currently tested
on 32 and 64 bit environments including Linux, Mac, Windows &
various other systems such as Plan 9, VMS, IoT devices, etc.)

@item Very little added to the standard C syntax (very easy to learn for
anyone familiar with C)

@item May be intermixed and linked with regular C or C++ code (easy to
incrementally add OO features to pre-existing apps)

@item Multiple Inheritance

@item True Dynamic Binding (polymorphism) Using Generic Functions

@item Fast, Cached, Method Lookup

@item Full Metaclass System (runtime class representation)

@item Enforced Encapsulation -- very important for large 
applications!  (No longer place 50\% of your code in include files, as
in C++. If something changes no need to recompile everything in
Dynace.  If a class needs to be changed you don't need to worry who is
looking under the hood, as in C++)

@item Modeled after CLOS providing the most flexible
solution without the overhead and runtime penalty, and large memory
requirements of an interpreter (as in CLOS and Smalltalk).

@item Compiles with a regular C or C++ compiler -- not an interpreter -- 
runs FAST!  (Doesn't have the runtime overhead and performance penalty
experienced with CLOS and Smalltalk)

@item Automatic Garbage Collection (system automatically frees unused
objects -- you no longer have to keep track of what and when things
must be freed)

@item Multiple Threads

@item Named & Unnamed Pipes

@item Named & Unnamed Counting Semaphores

@item Any Class located anywhere in the class hierarchy may be changed without
necessitating the re-compilation of any other module so long as the
pre-existing method protocols remain unchanged. (No need to re-compile
the world because you're not sure who is using or subclassing it -- A
major problem with C++ !!)

@item Uniform representation of all objects including Classes, Instances,
Methods, and Generics.  This allows easy construction of generic container
classes.

@item Powerful base class library including fundamental data types and
collection classes such as Dictionary and Double Linked Lists.

@end itemize


@page
@section Reasons to use Dynace
The following lists several reasons to use Dynace:

@enumerate
@item
Drastically reduce application complexity and increase its
maintainability by enforcing strong encapsulation -- no more need to
wonder which other modules are modifying your instance variables or
what effect changes in one module will have on other modules.

@item
Drastically reduce the need for recompilation since all class information
is defined in the source file (*.c) instead of header files (*.h).  Most
changes to a class, including its place in the class hierarchy, instance
or class variable declarations, new or changed methods (member functions),
have no compile time effect on modules which use, subclass or inherit 
from the changed class.

@item
Easily build generic functionality (like generic container classes)
which translates into increased use of existing code since Dynace
treats all objects, including classes, in a uniform manner and supports
weak and runtime object typing.

@item
Dynace is the easiest OO language to learn since it's almost entirely
normal C syntax, yet offers features far surpassing all the alternatives
such as true dynamic binding using generic functions, multiple inheritance,
fully metaclass based (runtime class representation) from the ground up
(like CLOS or Smalltalk), garbage collector (system automatically frees
unused objects), threader, and a powerful class library.  Defining classes
in Dynace is simple, all of the complexity and house cleaning chores are
handled by the Dynace pre-processor (dpp).

@item
Dynace applications run fast because they are compiled by your normal
C compiler into fast machine code (no interpreter or p-code).  Dynace
includes a pre-processor which converts Dynace class definition files
(which contain mostly straight C code) into plain C source files.

@item
It's easy to incrementally add OO facilities into pre-existing C code
because Dynace is just more C code.

@item
No longer be the victim of a vendor -- Dynace comes with full, portable
source code.

@item
You no longer have to distribute header files containing your proprietary
class structure in order for third parties to be able to use and
subclass your class.  Dynace doesn't need any of this information.

@item
Dynace OO model is much more flexible due to its metaobject base.  Now
you can actually have greater control over classes as well as instances.

@item
Dynace applications are royalty free (with the appropriate license).

@end enumerate


@page
@section Dynace vs. Other Languages

@subsection A comparison of Dynace and C++

The Dynace object oriented extension to the C language was created in
order to solve many of the problems associated with C++ while retaining
more backward compatibility with C than C++ and offering stronger object
oriented facilities.  This section discusses many of the shortcomings
of C++ and how Dynace address them, as well as other important facts
regarding Dynace.

@subsubsection Code In Header Files

C++ greatly encourages the movement of code and class definitions from
source files (.c) to header files (.h).  Because of its design, C++
requires this in order to support many of its compile time facilities.
However, while enabling many compile time facilities, it is also the
source of many of C++'s shortcomings.  Dynace is able to offer similar
compile time facilities (argument checking, ability to subclass a
pre-existing class, etc.) while greatly encouraging movement of code
from header files (.h) to source files (.c).  In fact, all Dynace
class definitions are entirely contained within source files.  The
consequences of these fundamental design differences will ripple
throughout this comparison.

C++ requires class definitions to be placed in header files and
included by other source files for three main reasons.  The first reason
is so that when other files need to declare or create new instances of
a given class, C++ needs to know the size of the instance at compile
time.

The second reason is that C++ needs to know the entire structure of
classes at compile time in order to create a new subclasses of them.
The third reason is that C++ needs to know at compile time the names
and locations of all externally accessible instance members (through
public, protected and friend members).

Dynace is able to perform these facilities with an increased level
of abstraction and code protection without any class definitions
in header files.

@subsubsection Frequency Of Re-compiles

One very important consequence of placing class definitions in header
files is that even very small changes in a class's definition
necessitates the re-compilation of source files which implement the
class, files which use the class and files which subclass that class.
On large, real-life applications this can amount to several hours
of time.  And for applications in which a lot of development is
being done, this may be required every thirty minutes.  The net
effect of this problem is that compile time quickly becomes the
place where most of the development time is spent.

Dynace, on the other hand, places all class definition code in
source (.c) files.  With Dynace, you can have a massive application,
take a class out of the middle of the class hierarchy and totally
modify it, and so long as the pre-existing member functions (methods)
retain their interface protocol (take the same arguments), only the
one source file which implements that class would have to be
re-compiled.  Even if the member function's syntax changed, the only
thing that would have to be modified and re-compiled were those
modules who used the changed member function -- something which would
have to be done in any environment.  The net effect of Dynace's
approach is that only the absolute minimum amount of time is wasted
on compiles.

@subsubsection Application Maintainability / Object Encapsulation

One of the principal features of the object oriented paradigm is
encapsulation.  Encapsulation is the ability to combine code and data
into a package such that the only external access provided is through
a well defined interface.  There are several important benefits associated
with strong encapsulation.

At development time, strong encapsulation helps organize the design and
breaks up the code into manageable size chunks.  At debug time, strong
encapsulation significantly limits the amount of code a programmer
has to look at in order to locate the problem.  After development,
strong encapsulation makes code easier to understand by new programmers
and limits the amount of code which must be checked and modified when
a given module is changed.  Strong encapsulation can have a major
effect on the development, debug, and enhancement times associated
with a project, not to mention reliability of the code.

C++ supports four types of encapsulation (protection) associated with
members of a class: private, protected, public and friends.  Of the
four types, three of them (public, protected and friends) are ways
to circumvent C++'s encapsulation features.  And private members are
only private from a narrow standpoint since it is declared in a
publicly accessible header file.

Note that from a technical standpoint, there is never a reason for
protected, public or friend members.  Anything you could ever need
to do can be done with private members and access functions.  The
net effect is that C++, for no apparent technical reason, does
not encourage strong encapsulation, nor does it even have the facilities
to support it.

Dynace, on the other hand, enforces strong encapsulation (private members)
and places all member definitions in a source (.c) file.  With this
approach, all the benefits of strong encapsulation may be maximally
achieved.


@subsubsection C++ Templates

In order to write similar code for many classes C++ provides
templates.  Templates are an extremely poor attempt to patch some
of the blatant shortcomings of C++.  There are two principal problems
with templates.  First, it suffers from the same compile-time limitations
as all the rest of C++, namely, templates only work on classes which
were set up at compile-time.  Any time you want to add a new class,
rebuild time.  The second problem is unnecessary duplication of code.
If you compile a template for five classes, you get five copies of the
same code!

Dynace has no templates and no need for them.  Dynace methods support any
class at runtime.  No need to re-compile the existing ones or to
duplicate any code.  The existing code adapts to new classes at
runtime!

@subsubsection Virtual Functions

Virtual functions give C++ a great deal of flexibility.  However,
there is a major problem associated with them.  If you purchase a
third party library and the vendor chooses not to implement a
particular function as a virtual function and you later need to use it
as a virtual function, it can't be done without the source code.

In Dynace all methods operate like C++ virtual member functions.  There
is never a one you can't use like a virtual function.  Dynace implements
these methods in a memory efficient manner so it's not prohibitive.

@subsubsection Learning Curve / Code Complexity

C++ is a large and complex language from a syntactical standpoint,
especially when compared to the minimalist C language.  There are two
problems associated with this complexity.  The first is that the
learning curve associated with going from a proficient C programmer to
a proficient C++ programmer is at least six months for most
programmers.  The second problem is that code written in C++ tends
to be more complex and therefore harder for new programmers to
understand and maintain.  The bottom line is that C++ development
tends to be much more expensive than C and employers end up being
much more dependent on a few ``experts''.

Dynace, on the other hand, uses straight C syntax and only adds a few,
easy to learn, constructs.  The learning curve associated with Dynace
is about one week (although you can actually use it the first day).
Pre-existing C code can co-exist with Dynace code better than C++.
And due to the simpler syntax and strong encapsulation features of
Dynace, it is much easier for new programmers to pick up Dynace and
the new application.  This translates into real dollars saved.

Dynace adds powerful capabilities of object oriented technology in a
simple, idealistic, theoretically pure way.  Dynace has taken the best
attributes of CLOS, Smalltalk and Objective--C and implemented it in a
way which is natural to a C programmer.  A Dynace program looks just
like any other C program.  Dynace is, therefore, much simpler and easier
to master than C++ and the resulting tool is more powerful and expressive.

@subsubsection Code Re-usability

C++ is a strongly typed language.  This means that all type information
associated with variables and function arguments must be completely
specified at compile time.  While attempting to create reusable code
a programmer must create routines which are general purpose and can
therefore be used in a variety of situations.  The problem is that
these two points are at odds with each other.

When creating general purpose (reusable) software components it is often
desirable to have the routine handle a variety of data types, as is
often needed in container classes.  Strongly typed languages require
that the potential data types be known at the compile time of the routine.
This means that every time a new data type is used the ``general purpose''
routine would have to be modified and recompiled.  How many books and
articles have you seen which attempt to trick C++ into handling data
in a generic fashion?  This important feature of good software development
(creating reusable software) is an up hill battle with C++.

Dynace, on the other hand, is a weakly typed language.  This means
that Dynace treats all objects (including classes) in a generic
fashion.  Creating generic (reusable) routines like container classes
are trivial in Dynace.  Dynace is also able to validate an object or
its type at runtime.


@subsubsection Strong vs. Weakly Typed Languages

Strongly typed languages (such as C++) require that the programmer
specify exactly what data types a particular routine may handle.
Whereas, weakly typed languages (such as Dynace, CLOS and Smalltalk)
allow a programmer to define a process in abstract terms without
specifying any particular data types.  It's kind of like the
difference between speaking literally and speaking abstractly.

If teaching a child were anything like programming a computer and you
could only teach him literally you would have to explain to him that
if he had two dogs and someone gave him another, he would then have
three dogs.  If he has two apples and someone gives him another he'll
have three apples.  If he has two trains and someone gives him another
he'll have three trains, and so on.  However, if you could teach your
child in abstract terms you could simply tell him that if he has two
of something and he receives another, he'll then have three of them.

The advocates of strongly typed languages like to point out that their
language can uncover an incorrect argument to a function at compile
time, before the program is even run.  The reality of it is that passing
the incorrect data type to a function is just one of many, many possible
types of errors a given program may contain.  Why restrict a language's
capabilities so severely to uncover just one type of error.  

In fact when programming in C++ it is quite easy to create a situation
where the correct type is being passed (a pointer to some structure)
but when the code gets run it hangs the system because it wasn't
initialized and the compiler can't check for that. So C++, as well as
other strongly typed languages, are quite good at performing static,
compile time, type checking at the expense of a flexible language.
And then when it comes to dynamic (runtime) checking you're on your
own.

While Dynace does check arguments at compile time, it does not attempt
to validate each argument beyond whether it's an object or a particular
C data type.  In other words, all Dynace objects are treated opaquely.
Not only can Dynace verify the type at runtime (essentially the same
test C++ performs at compile time), but it will also validate an
uninitialized pointer!  If there is an attempt to use an uninitialized
object or an incorrect type of object Dynace prints an error message.
No hung system.

Weakly typed languages, such as Dynace, provide an extremely more
flexible and powerful means of expressing computer algorithms, and
they do this with an increase in error detection and handling as
compared to strongly typed languages such as C++.

Weak typing is a much stronger approach to implementing the object
oriented model.  Weak typing allows much more powerful, flexible, and
general purpose solutions to be developed.  Since strong typing
requires the types of data a routine handles be known at compile time,
at the very least, the module will have to be recompiled if it is to
handle a new type of object, if not entirely re-coded.  Weak typing
allows the development of truly general purpose, reusable software
components.

@subsubsection Generic Container Classes

The tremendous difficulty involved in tricking C++ into implementing
generic container classes, such as dictionary, linked list and bags,
typifies the biggest problem with C++, lack of runtime flexibility.
How many books and computer magazine articles have to be written
in order to show how to trick C++ into accomplishing something
which should be fundamental and trivial.  It lies at the heart of
the value of object oriented concepts.

The properties of C++ which make is so inflexible are it's strong data
typing and lack of runtime class representation.

Creating generic container classes is fundamental and trivial in Dynace.
This is a very important difference.

@subsubsection Runtime class representation

C++ has no runtime representation of classes and no metaclasses.  In
fact, a C++ class is not an object at all.  In short, C++ is a half
hearted attempt to add object oriented capabilities to the C language.
Dynace implements all object oriented concepts in a pure, flexible fashion.
In fact, in Dynace, classes are objects just like any other.

Since C++ has no runtime class representation its compiler must know
the entire (supposed to be private) structure of every class the module
might use.  This causes two very significant problems.  First of all,
it encourages the programmer to put a great deal of code in the header
files.  The more code put in the header the greater the chance that
something in the header will have to be changed when modifying a class.
If a header file changes all programs which are dependent on this
header file must be re-compiled.  This is because the ``new'' operator
must know the class size at compile time (even if all the elements are
private!).  What this ends up causing, in a real life application
where there are many classes in a complex hierarchy, is that the
entire application must be totally re-compiled every time there is the
smallest modification (in order to play it safe).  This one factor can
multiply the time it takes to develop an application.

The second problem with not having a runtime class representation is a
lack of flexibility and a lack of the ability to create truly general
purpose routines.  Since C++ does all its type checking at compile time
it makes sense that at runtime it can only handle the types it was
compiled for.  Therefore, if you compile some general purpose routines
into a library (or worse yet, buy a third party's library without source)
and later wish to use its existing functionality, but for a class it was
not compiled for, you would have to make some sort of code changes and
re-compile the library (if you have source).

Since Dynace has a runtime class representation it doesn't have any of
these problems.  All code and structure definitions for a class reside
in the .c file.  There is no need to put anything in a header.  If a
class changes it has no compile-time effect on any other modules.  As
long as its original interface remains the same (although at can be
added to) not a single line of code in any other module would have to
be changed, no other modules would have to be re-compiled, and no
libraries rebuilt.

Again, since Dynace has a runtime representation truly general purpose
routines can be build without requiring the old code to be
re-compiled.  You can build a general purpose routine, put it in a
library, and then two years later create a totally new class, and run
an instance of the new class against the old code in the library
without ever having to re-compile the old code.

@subsubsection Efficiency


Dynace is not an interpretive language. Dynace programs are compiled with a
standard C compiler.  The majority of the code is just standard compiled
C code with no performance penalty.  The only place Dynace incurs a runtime
cost is at the point of method dispatch.  Since C++ also incurs a runtime
cost when using virtual functions, there is not much of a difference
between the performance of Dynace programs when compared to C++.  In addition,
Dynace gives the ability to locally cache a method pointer in tight loops so
as to totally dispense with the runtime overhead.


@subsubsection Compatibility With C

C++ is somewhat compatible with C, however, Dynace is entirely
compatible with C.  Code which uses Dynace classes is just plain C
code.  Code which defines new Dynace classes consists of standard C
with a few added constructs to define classes and methods.  This then
goes through a Dynace preprocessor which emits straight C.

The two main points about this is that Dynace is more compatible with
C than C++, and it's easier to incrementally add object oriented
facilities to a pre-existing C application with Dynace.


@subsubsection Garbage Collection

While C++ will automatically dispose of some types of objects (automatic
variables), it can not automatically dispose of those objects allocated
from the heap.  In a complex application most objects would be allocated
from the heap, requiring explicit disposal in C++.  Dynace's garbage collector
handles all object reclamation automatically.


@subsubsection Class Library

C++ environments come with an absolute bare minimum of fundamental classes.
Of course there are many C++ class libraries available, however, given
the fundamental architecture of C++, these libraries are typically
incompatible and difficult to understand, use and extend.

Dynace comes with a complete set of fundamental classes including classes
to represent all the basic C types, a variety of container classes (sets,
dictionaries, linked lists, associations, etc.), multi-dimensional dynamic
arrays, threads, pipes and semaphores.


@subsubsection Conclusion


Dynace adds powerful object oriented capabilities in a simple,
idealistic, theoretically pure way.  Dynace has taken the best
attributes of CLOS, Smalltalk and Objective--C and implemented it in a
way which is natural to a C programmer.  Dynace has no special syntax
and is easy to learn.  A Dynace program looks just like any other C
program.

Dynace offers effective solutions to a variety of problems associated
with software development, including reducing development time by
segregating and organizing various application components, providing a
powerful class library, making pre-written components maximally
reusable, reducing language and application learning curves for new
programmers, and making an application easy to debug, maintain and
enhance.



@subsection A comparison of Dynace and Smalltalk


Dynace is a very dynamic and flexible language just like
Smalltalk.  Dynace and Smalltalk share the following features:

@itemize @bullet
@item
Both are weakly typed languages and therefore allow
for the development of powerful routines which provide truly generic
and re-usable software components.

@item
Both provide true dynamic binding on all messages.

@item
Both provide tight encapsulation of objects for
easy program maintenance.

@item
Both provide a similar class hierarchy including
the Object, Behavior, Class and MetaClass classes.

@item
Both provide a complete metaclass system which allows
all objects in the system to be treated uniformly.

@item
Both provide the ability to perform automatic reclamation
of unneeded objects (garbage collection).

@item
Both provide the ability to create, synchronize and
pass information between multiple independent processes (threads).

@end itemize


The following lists some of the differences between Dynace and
Smalltalk:

@itemize @bullet
@item
Dynace uses a standard C syntax which is easy for C programmers to learn.

@item
Dynace may be freely mixed with regular C code and may also link with
pre-existing C libraries.

@item
Dynace uses a standard C compiler and generates stand alone native
executable code.  Smalltalk typically generates code for a virtual
machine which must be included or bound with an application.

@item
Since Dynace generates native code it runs many times faster than Smalltalk
and requires much, much less RAM (just a little more than a typical C
program).

@item
Dynace is written in standard C and comes with full source code.  It is
therefore very portable and drastically reduces your dependence on a
vendor.

@item
Dynace supports multiple inheritance.

@end itemize

@section Obtaining The System
The entire system is available at @uref{https://github.com/blakemcbride/Dynace}





@page
@section Contents
Once the system has been installed there will exist a series of
directories under the @code{Dynace} root directory.  The system will contain
the following directories:


@table @asis
@item lib
This is the location of all the Dynace libraries.  You may wish to add
this directory to the list of paths your linker searches.

@item include
This is the location of the include files necessary to compile Dynace
applications.  You may need to add this directory to the path your
compiler uses to search include files.

@item bin
Executable files necessary for development with the Dynace system.
This directory should be added to your normal search path for executable
programs.

@item examples
Example programs used to learn & demonstrate the Dynace object oriented
extension to C.

@item docs
Misc. documentation files.

@item manual
The Dynace manual.

@item kernel
Complete source to the Dynace kernel.

@item class
Source to all of the Dynace base classes.

@item threads
Complete source to the multi-threader, pipes and semaphores.

@item generics
Files necessary to build the system generics files from scratch.

@item dpp
Complete source to the @code{dpp} utility.

@item utils
Complete source to the utility programs.

@end table



The only files that are absolutely necessary for a developer are those
located in the @code{lib, include} and @code{bin}
directories.




@page
@section Learning The System
This manual contains a description and rationale of object oriented
concepts, a detailed description of the Dynace system and complete
reference to all Dynace classes including examples.  The example programs
included with Dynace provide a step-by-step tutorial to getting started.

A user of the Dynace system will need to know the C language.  However,
no previous experience with object oriented concepts is necessary.

The best approach to learning Dynace would be to start by reading
chapters 1 (Introduction) and 2 (Concepts).  Then read sections 3.1 and
3.2 (Using Dynace Classes, including its sub-sections) while working
through the example programs.  When the example programs start creating
their own classes follow with section 3.3 (Defining Dynace Classes,
including its sub-sections).  Ultimately it is best to read the entire
manual.

If you are somewhat familiar with object oriented concepts and wish
to dive right in and get a feel for the system you may go directly
to the example programs after reading this chapter.

While working through the examples you may refer to the index (located
in the back of the manual) to locate information on all Dynace methods,
macros, variables, functions and data types.


It is highly recommended that a programmer already familiar with object
oriented concepts still read the entire @emph{Concepts} chapter because
of its particular relevance to the approach taken in the Dynace system.







@page
@section Quick Start & Tutorial
The Dynace system includes a series of examples which serve both as a
Quick Start and a Tutorial.  Each example is contained in its entirety
in an independent directory.  This is done in order to illustrate the
exact files and steps necessary to create a single application.

The example programs are contained in sub-directories under the
@emph{examples} directory and are named @emph{examNN} (where the
@emph{NN} is a two digit number).  These numbers are significant in that
they describe the correct order that the examples should be followed.
Each example depends on knowledge built up in previous examples which is
not repeated.

Each example contains a @emph{readme} file which describes information
relating to the purpose of the example and build instructions.  These
files should be read first.  Each example also includes makefiles.
The source to the examples is also included.

The example programs are not intended as a substitute for the manual.
They are simply meant to augment the manual and accelerate the initial
learning curve.

The following is a list of the enclosed examples:

@table @asis
@item 01
Illustrates the steps necessary to compile and link a program which
incorporates and correctly initializes Dynace.

@item 02
Illustrates the creation, use and disposal of a simple object.

@item 03
Illustrates the creation, use and disposal of more simple objects.

@item 04
Illustrates the creation, use and disposal of an instance of the
LinkObject class.

@item 05
Illustrates the use of the LinkObjectSequence class to enumerate through
the elements of a linked list.

@item 06
Illustrates the creation, use and disposal of an instance of the
StringDictionary class.

@item 07
Illustrates the process of enumerating through the elements of a
set or dictionary using the SetSequence class.

@item 08
Illustrates how Dynace handles errors.

@item 09
Illustrates the value of and initialization procedure for the automatic
garbage collector.

@item 10
Illustrates the creation and initialization of a new class.

@item 11
Illustrates the creation of a new method and generic function.

@item 12
Illustrates additional points about methods and generics.

@item 13
Illustrates the independence one instance has from another.

@item 14
Illustrates use of class variables / methods using gNew / gDispose.

@item 15
Illustrates initialization of instance variables with gNew.

@item 16
Another illustration of initialization of instance variables with gNew.

@item 17
Illustrates subclassing.

@item 20
Illustrates the use of threads.

@item 21
Illustrates the use of the BTree classes.

@item 30
Illustrates the process of getting information from the system.

@item 31
Illustrates how generic functions are first class C objects.

@item 32
Illustrates how to locally cache a method lookup and avoid the
runtime cost.

@item 33
Illustrates most of the methods associated with the String class.

@item 34
Illustrates some of the numeric and date formatting abilities.

@end table


See the file @code{examples/list} for the most up-to-date list
of example programs.


@page
@section Manual Organization
This manual serves as both a user manual and a complete reference manual
to the Dynace system.  It makes no assumptions about the users knowledge
of object oriented programming.  It does, however, assume a working
knowledge of the C language.

Chapter 1 (Introduction) covers background material needed to orient
a new user.

Chapter 2 (Concepts) covers object oriented concepts both in an abstract
sense and as it relates to the Dynace system.  It does this without
introducing very much syntax or other mechanics.

A full understanding of these concepts is crucial to the effective use
of the Dynace system.  Even if you have experience with other object
oriented languages it is important to understand these concepts as they
are implemented within Dynace.  Note, however, that it is not necessary to
know everything before effective use of Dynace can be accomplished.  Full
understanding will come with use of the system.

Chapter 3 (Mechanics) introduces the exact syntax and procedures
necessary to use the Dynace system.  It does much of this with examples.
Although this chapter does not make explicit reference to the Dynace example
programs, they should be reviewed while reading this chapter.

Chapter 4 (Kernel Reference) provides a detailed reference to all
classes, methods, macros, functions and data types associated with
the Dynace kernel.

Chapter 5 (Class Reference)  provides a detailed reference to all
classes and methods associated with the class library included
with the Dynace system.

The index (located in the back) provides a complete alphabetical
listing of all classes, methods, macros, functions and data types
described in chapters 4 and 5.






@page
@section Contact Information

@table @asis

@item Email
blake@@mcbridemail.com

@item Discussion
There is a Dynace discussion group at 
@iftex
@hfil@break
@end iftex
https://github.com/blakemcbride/Dynace/discussions

@end table




@section Use, Copyrights & Trademarks
Copyright  @copyright{} 1996 Blake McBride
All rights reserved.

Redistribution and use in source and binary forms, with or without
modification, are permitted provided that the following conditions are
met:

1. Redistributions of source code must retain the above copyright
notice, this list of conditions and the following disclaimer.

2. Redistributions in binary form must reproduce the above copyright
notice, this list of conditions and the following disclaimer in the
documentation and/or other materials provided with the distribution.

THIS SOFTWARE IS PROVIDED BY THE COPYRIGHT HOLDERS AND CONTRIBUTORS
``AS IS'' AND ANY EXPRESS OR IMPLIED WARRANTIES, INCLUDING, BUT NOT
LIMITED TO, THE IMPLIED WARRANTIES OF MERCHANTABILITY AND FITNESS FOR
A PARTICULAR PURPOSE ARE DISCLAIMED. IN NO EVENT SHALL THE COPYRIGHT
HOLDER OR CONTRIBUTORS BE LIABLE FOR ANY DIRECT, INDIRECT, INCIDENTAL,
SPECIAL, EXEMPLARY, OR CONSEQUENTIAL DAMAGES (INCLUDING, BUT NOT
LIMITED TO, PROCUREMENT OF SUBSTITUTE GOODS OR SERVICES; LOSS OF USE,
DATA, OR PROFITS; OR BUSINESS INTERRUPTION) HOWEVER CAUSED AND ON ANY
THEORY OF LIABILITY, WHETHER IN CONTRACT, STRICT LIABILITY, OR TORT
(INCLUDING NEGLIGENCE OR OTHERWISE) ARISING IN ANY WAY OUT OF THE USE
OF THIS SOFTWARE, EVEN IF ADVISED OF THE POSSIBILITY OF SUCH DAMAGE.

@section Credits


The Dynace Object Oriented Extension to C
and its associated documentation were written by Blake McBride
(blake@@mcbridemail.com).

Credit and great appreciation is given to several sources for the
ideas and inspiration behind many of Dynace's facilities as follows.

John Wainwright for the development and sharing of ideas concerning
early releases of his Objects In C (OIC) system.  He is an extremely
talented individual with many unique and innovative approaches to
software development issues.  I am proud to call him my friend.

Gregor Kiczales, Jim des Rivi@`eres, and Daniel G. Bobrow for their
book ``The Art of the Metaobject Protocol'' and their CLOS
system.

Adele Goldberg and David Robson for their books on Smalltalk-80 and
the Xerox Palo Alto Research Center for providing access to much
powerful and innovative technology.

And finally, Dr.@ Brad Cox for pioneering object oriented extensions
to C with his Objective--C system.


