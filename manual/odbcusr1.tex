@c -*-texinfo-*-

@chapter ODBC Tutorial

@section Opening and Closing a Database

Prior to any use of a database, it must be opened.  Two functions are
available to accommodate the function as follows:

@deffn {OpenDatabase} OpenDatabase::Database
@sp 2
@example
@group
DB = gOpenDatabase(Database, source, id, pw)

object  DB;             /*  database object  */
char    *source;        /*  data source name as defined with
                            the ODBC administrator  */
char    *id;            /*  the user id to log into the
                            database with  */
char    *pw;            /*  the password associated with
                            the user id   */
@end group
@end example
This will open the specified database using the specified user id and password.
@code{NULL} is returned of the database cannot be opened.
@end deffn

@deffn {QueryDatabase} QueryDatabase::Database
@sp 2
@example
@group
DB = gQueryDatabase(Database, wind)

object  DB;             /*  database object  */
object  wind;           /*  parent window object or NULL  */
@end group
@end example
This will query the user to select a database and then open the specified database
using the user ID and password entered using the ones configured with the ODBC administrator.
@end deffn

The following method is used to close a database.

@deffn {Dispose}  Dispose::Database
@sp 2
@example
@group
DB = gDispose(DB)

object  DB;             /*  database object  */
@end group
@end example
This procedure will close the database and @code{NULL} out the @code{DB} variable.
@end deffn




@section Statements

@subsection Introduction

One concept which is important to fully understand is the
@emph{Statement} object.  A Statement object allows the program a means
to execute an SQL command and then have access to the data returned by
that particular command.  Once a command's data is no longer needed, the
Statement object may be re-used for other (and arbitrary) SQL commands.
A single database may have numerous Statement objects associated with it
at any given point.  This number is only limited by the particular
database vendor.

@subsection Creating and Disposing of Statement Objects


@deffn {NewStatement}  NewStatement::Database
@sp 2
@example
@group
stmt = gNewStatement(DB);

object  DB;     /*  the database object      */
object  stmt;   /*  a new statement object   */
@end group
@end example
This will create a new statement object;
@end deffn



@deffn {Dispose}  Dispose::Statement
@sp 2
@example
@group
gDispose(stmt);

object  stmt;   /*  the statement object   */
@end group
@end example
This will dispose of the statement object;
@end deffn



@section Adding Records

@subsection Overview

Adding of records is accomplished in three steps as follows.

{@leftskip .5in
@enumerate
@item
Select the file or table in which records are to be added.  This only
needs to be done once even if multiple records are to be added.

@item
Assign the data to the fields.

@item
Add the record.
@end enumerate
}

Once a record is added, additional records may be added by looping through
steps 2 and 3.  Data assigned to a field which is invariant need not be
re-assigned.

@subsection Selecting the Table

Selecting the table can be accomplished in two ways.  The most common way would be
via the @code{gInsert} method.  Alternatively, if you are already in the middle of a select
command in which all the fields are selected in the desired table and there are no
joins, as follows:

@example
gDBSelect(stmt, "select * from mytable");
@end example

you need not perform any additional function.  The table is already selected.

@deffn {Insert}  Insert::Statement
@sp 2
@example
@group
r = gInsert(stmt, table);

object  stmt;   /*  the statement object  */
char    *table; /*  the table to use      */
int     r;      /*  0 on success          */
@end group
@end example
This method selects a table for subsequent record adds.
@end deffn


@subsection Assigning Field Data

Once the desired table is selected, field data may be assigned using the
following methods.

@example
@group
stmt = gFldSetString(stmt, fld, str);

stmt = gFldSetChar(stmt, fld, ival);

stmt = gFldSetShort(stmt, fld, ival);

stmt = gFldSetUnsignedShort(stmt, fld, uival);

stmt = gFldSetLong(stmt, fld, lval);

stmt = gFldSetDouble(stmt, fld, dval);


object  stmt;   /*  the statement to operate on  */
char    *fld;   /*  the field name               */
int     ival;   /*  an integer value             */
unsigned int uival;  /*  unsigned value          */
long    lval;   /*  long value                   */
double  dval;   /*  double value                 */
@end group
@end example

The object returned is simply the statement object passed.  Any attempt
to use a non-existing field will cause an error in which the application
will be terminated.  If you assign a value of the wrong type, the system
will convert it for you.  Date fields are simply converted to @code{long}'s
of the form @code{YYYYMMDD}.

@subsection Adding The Record

Once the data has been assigned the record may be added as follows.

@deffn {AddRecord}  AddRecord::Statement
@sp 2
@example
@group
r = gAddRecord(stmt);

object  stmt;   /*  the statement object  */
int     r;      /*  result                */
@end group
@end example
This will cause the data to be added to the table.  Zero is returned
upon success, other values indicate an error.  Most errors are trapped
and cause the application to abort, however.
@end deffn

@section Reading Records

@subsection Overview

Reading records is accomplished in three steps as follows.

{@leftskip .5in
@enumerate
@item
Select a set of records from which to access data.

@item
Select a particular record.

@item
Access the field data.
@end enumerate
}

@subsection Selecting a Record Set

Selecting a record set is accomplished by executing a normal SQL
@code{select} command as follows.

@deffn {DBSelect}  DBSelect::Statement
@sp 2
@example
@group
r = gDBSelect(stmt, cmd);

object  stmt;   /*  the statement object    */
char    *cmd;   /*  the SQL select command  */
int     r;      /*  result                  */
@end group
@end example
This method will execute an arbitrary SQL select command given in @code{cmd}.  This command may contain joins,
order by, and where clauses as well as any other legal SQL select command. zero is returned upon success.  For example:  
@example
gDBSelect(stmt, "select * from myfile");
@end example
@end deffn

@deffn {DBSelectOne}  DBSelectOne::Statement
@sp 2
@example
@group
r = gDBSelectOne(stmt, cmd);

object  stmt;   /*  the statement object    */
char    *cmd;   /*  the SQL select command  */
int     r;      /*  result                  */
@end group
@end example
This method is a convenience for the case when a single specific record
is desired.  When this method is used there is no need to also select a
specific record.  Data fields may be accessed immediately.  zero is
returned upon success.  
@end deffn


@subsection Selecting a Record

The following functions allow the selection of a specific record within a
selected set.  Note that if @code{gDBSelectOne} is used, none of these
functions should be used.

@example
@group
r = gFirstRecord(stmt);

r = gLastRecord(stmt);

r = gNextRecord(stmt);

r = gPrevRecord(stmt);

r = gSetAbsPos(stmt, pos);

r = gSetRelPos(stmt, pos);

object  stmt;   /*  the statement object           */
long    pos;    /*  relative or absolute position  */
int     r;      /*  0=success                      */
@end group
@end example


@subsection Accessing Field Data

The following functions are used to access field data.

@example
@group
sval = gFldGetString(stmt, fld);

cval = gFldGetChar(stmt, fld);

sval = gFldGetShort(stmt, fld);

uval = gFldGetUnsignedShort(stmt, fld);

lval = gFldGetLong(stmt, fld);

dval = gFldGetDouble(stmt, fld);

oval = gFldGetValue(stmt, fld);

object  stmt;   /*  the statement object  */
char    *fld;   /*  the field name        */
short   sval;   /*  return values         */
char    cval;
unsigned short uval;
long    lval;
double  dval;
object  oval;
@end group
@end example

If the field type is not the same as that requested, the system will
convert it for you.


@section Changing and Deleting Records

The procedure for changing or deleting records is similar to that of
reading records.  Steps 1 and 2 are the same.  In addition, once a
record had been read it may be immediately changed or deleted without the
need to re-execute the first two steps.  Step 3 is as follows.

@deffn {UpdateRecord}  UpdateRecord::Statement
@sp 2
@example
@group
r = gUpdateRecord(stmt);
        or
r = gDeleteRecord(stmt);

object  stmt;   /*  the statement object  */
int     r;      /*  result (0=success)    */
@end group
@end example
This method will either update or delete the last selected record.
@end deffn


@section Executing SQL Commands

The following methods allow the execution of arbitrary SQL commands.

@deffn {Execute}  Execute
@sp 2
@example
@group
r = gExecute(DB, cmd);
     or
r = gExecute(stmt, cmd);

object  DB;     /*  database object    */
object  stmt;   /*  statement object   */
char    *cmd;   /*  SQL command        */
int     r;      /*  result (0=success) */
@end group
@end example
These methods allow for the execution of arbitrary SQL command when no
specific data access is required (you aren't trying to access field
values).
@end deffn


@deffn {ExecuteFile}  ExecuteFile
@sp 2
@example
@group
r = gExecuteFile(DB, file);
       or
r = gExecuteFile(stmt, file);

object  DB;     /*  database object    */
object  stmt;   /*  statement object   */
char    *file;  /*  file name          */
int     r;      /*  result (0=success) */
@end group
@end example
The method will cause all the SQL commands in a file to be executed against database @code{DB}.
A file extension of @code{.sql} is assumed.
@end deffn


@section Dropping Tables



@deffn {DropTable}  DropTable::Database
@sp 2
@example
@group
r = gDropTable(DB, tbl);

object  DB;     /*  database object    */
char    *tbl ;  /*  table              */
int     r;      /*  result (0=success) */
@end group
@end example
The method will drop table @code{tbl}.  If the table didn't already
exist this command will be ignored.
@end deffn



@section Clearing Fields


@deffn {Clear}  Clear::Statement
@sp 2
@example
@group
r = gClear(stmt);

object  stmt;   /*  statement object   */
int     r;      /*  result (0=success) */
@end group
@end example
This method will clear (zero out) all the fields.  It may only be used
after a select command.  It is often used after some records have been
read and a new record is to be added without an intervening record set
selection.
@end deffn


@section WDS Interface

The following methods are used to associate ODBC data fields to dialog controls.


@deffn {AttachDialog}  AttachDialog::Statement
@sp 2
@example
@group
stmt = gAttachDialog(stmt, dlg);

object  stmt;   /*  statement object   */
object  dlg;    /*  a dialog object    */
@end group
@end example
This method is used to associate all controls within a dialog to fields
associated with a statement in which the fields and controls have the
same name.  Once this association is formed subsequent use of the dialog
will cause the data to be entered into the associated data fields.
This method should only be used subsequent to a @code{gDBSelect}, @code{gDBSelectOne}, or
a @code{gInsert}.
@end deffn


@deffn {AssociateCtl}  AssociateCtl::Statement
@sp 2
@example
@group
stmt = gAssociateCtl(stmt, fld, ctl);

object  stmt;   /*  statement object   */
char    *fld;   /*  a database field   */
object  ctl;    /*  a control object   */
@end group
@end example
This method is used to associate control @code{ctl} with database field @code{fld}.
Once this association is formed subsequent use of the control
will cause the data to be entered into the associated data field.
This method should only be used subsequent to a @code{gDBSelect}, @code{gDBSelectOne}, or
a @code{gInsert}.
@end deffn
