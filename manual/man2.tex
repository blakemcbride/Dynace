@c -*-texinfo-*-

@c  Copyright (c) 1996 Blake McBride
@c  All rights reserved.
@c
@c  Redistribution and use in source and binary forms, with or without
@c  modification, are permitted provided that the following conditions are
@c  met:
@c
@c  1. Redistributions of source code must retain the above copyright
@c  notice, this list of conditions and the following disclaimer.
@c
@c  2. Redistributions in binary form must reproduce the above copyright
@c  notice, this list of conditions and the following disclaimer in the
@c  documentation and/or other materials provided with the distribution.
@c
@c  THIS SOFTWARE IS PROVIDED BY THE COPYRIGHT HOLDERS AND CONTRIBUTORS
@c  "AS IS" AND ANY EXPRESS OR IMPLIED WARRANTIES, INCLUDING, BUT NOT
@c  LIMITED TO, THE IMPLIED WARRANTIES OF MERCHANTABILITY AND FITNESS FOR
@c  A PARTICULAR PURPOSE ARE DISCLAIMED. IN NO EVENT SHALL THE COPYRIGHT
@c  HOLDER OR CONTRIBUTORS BE LIABLE FOR ANY DIRECT, INDIRECT, INCIDENTAL,
@c  SPECIAL, EXEMPLARY, OR CONSEQUENTIAL DAMAGES (INCLUDING, BUT NOT
@c  LIMITED TO, PROCUREMENT OF SUBSTITUTE GOODS OR SERVICES; LOSS OF USE,
@c  DATA, OR PROFITS; OR BUSINESS INTERRUPTION) HOWEVER CAUSED AND ON ANY
@c  THEORY OF LIABILITY, WHETHER IN CONTRACT, STRICT LIABILITY, OR TORT
@c  (INCLUDING NEGLIGENCE OR OTHERWISE) ARISING IN ANY WAY OUT OF THE USE
@c  OF THIS SOFTWARE, EVEN IF ADVISED OF THE POSSIBILITY OF SUCH DAMAGE.

@chapter Concepts
Over time, the methods used in the development of computer software has
evolved from a @emph{Goto} model, to a @emph{Structured Programming}
model, and finally the @emph{Object Oriented model}.  In the
beginning, the @code{goto} model was able to accommodate all the needs of the
programmers.  However, as the size of programs became larger and
more complex, the goto code began to look like spaghetti.  No one, except
perhaps the original author, was able to understand, and therefore
maintain, a complex program which was jumping all over the place.  The
solution to this problem was the Structured Programming model.

The Structured Programming model got rid of @code{goto} statements and
replaced them with the more structured @code{for} and @code{while}
statements.  With this model, the programmer knows the beginning and
ending location of the loop or conditional, as well as what the specific
conditions are for executing the contained code.  No more spaghetti.

Structured Programming also stressed the use of subroutines.  This
allowed blocks of code to be packaged in a form which could be easier
understood as a small, well defined, unit.  It also enabled, to some
degree, the ability to create reusable code.  It should be noted that
while packaging blocks of code as subroutines greatly increased the
programmers ability to manage large programs, it also comes at a cost.
Calling blocks of code as subroutines is substantially more expensive,
in terms of computer cycles, than a @code{goto} statement.  As time has
progressed, however, it has generally been accepted that the benefits of
subroutines far exceed the costs.

The Structured Programming model worked well for many years.  However,
as computer programs became still larger, the programmers ability to
manage larger, more complex projects became increasingly difficult.  The
ability to use common subroutines across varying applications was very
limited.  It became very difficult to manage the relationship between
all the varying data structures and various subroutines.

Three things were needed to solve the problems of the Structured Programming
model.  First, there needed to be a way of isolating or packaging related
groups of code and data, so that if a modification had to be made, or a bug 
fixed, only a specific group of code and data had to be understood and
modified.  This would allow a programmer to change a system, or add new
functionality, without first having to understand all the intricacies of the
entire system.

Second, any changes made to the related group of code and data should
have minimal effect on any other code or data.  This way, if a
programmer changes a module, he shouldn't then have to search every
place it's used to see what additional changes had to be made in order
to accommodate the module change.

The third problem is that of maximizing reusability.  For years
programmers have been writing the same linked list routines, the same
sort routines, and the same hash table routines over and over.  The
ability to create general purpose routines, which could be used in many
places, over and over, is of immense value.  After the creation of
a good general purpose library, applications development time would be
reduced considerably, and code reliability increased considerably.
In addition, any enhancements made to the general purpose library
would have an automatic benefit to all applications which use the library.

All three of the above issues are addressed in the Object Oriented
model.  The first two are addressed by the @emph{encapsulation, data
abstraction}, and @emph{loose coupling} attributes of the Object Oriented
model.  The third issue is addressed by the @emph{inheritance} and
@emph{dynamic binding} aspects of the Object Oriented model.  All these
issues, as well as others, will be addressed by the remainder of this
manual.

@page
@section Object Oriented

The Object Oriented model of programming is embodied by three
fundamental concepts, a bit of new terminology, and a few new
programming mechanisms used to implement the new concepts.  The three
fundamental concepts are @emph{encapsulation, polymorphism, and inheritance}.
These concepts will be introduced first, followed by the mechanisms.
The terminology will be introduced as it is encountered.

@subsection Encapsulation
Encapsulation allows the programmer to tightly bind a data structure
with the routines which access it, into an item called an @emph{object}.
Encapsulation reduces the ways in which outside data or procedures may
effect a given object.  Outside elements may only effect a given object
by a limited number of well defined methods.  Using a limited number of
ways to effect an object causes @emph{loose coupling}.  A well defined
method of effecting an object is a @emph{protocol}.

A limited example of this idea can be achieved, for example, by creating
a unique structure declaration inside a particular C source file, or
module, as opposed to inside an include file.  By doing this, only the
functions in the same source file as the structure declaration may
access elements of the structure, since those are the only routines
which have any knowledge of its structure.  Anything outside the
defining module must call a non-static function, located within the
module, in order to effect the data structure.  These non-static
functions would, in effect, comprise the modules @emph{protocol}.  The
module would have @emph{loose coupling} because the only way outside
routines may effect the internal data structure, or state of the object,
is, again, through the non-static functions.  External routines have no
access to the static, or local, routines, and have no direct access to
the data structure itself.

Further, a very important form of @emph{data abstraction} is achieved
through encapsulation. Data abstraction refers to the ability to change
the structure of data without having to change any external routines
which, indirectly, use the data structure.  As in the previous example,
the data structure, as well as the entire implementation of the module
may be changed entirely, and so long as the pre-existing protocol and
module functionality remains intact, all other modules which use this
module will continue to work without @emph{any} modifications.  Major
implementation details may change and be made more efficient, and
major additions to the module's functionality may be added, without
any effect to external, pre-existing, code.

As opposed to the above example, the form of encapsulation implemented
in most of the more advanced object oriented models, as well as Dynace, is
much tighter.  External routines not only have no access to the elements
of the data structure, but also have no direct access to @emph{any} of
the routines which may effect the data structure!  The data structure
and associated routines are tightly packaged into a unique object.  The
only way to effect the object is to request the object itself to make
any modifications to itself, or perform any operation.  The meaning and
methods to achieve this will become increasingly clear as we progress.


@subsection Polymorphism And Overloading
In a normal C program, you may only have one function with any particular
name.  For example, if you had a non-static function @code{fun1()} in one
module, you couldn't have another non-static function @code{fun1()} in
another module.  The two modules wouldn't link.  If the program did
link, and a third module called @code{fun1()}, the system wouldn't know
which one you wanted to execute.

@emph{Polymorphism} or @emph{overloading} refers to the ability of the
system to determine, based on context, which particular routine to
execute.  The context normally refers to the arguments to the routine.
So, for example, in our previous example, you would be able to link two
modules, each with non-local @code{fun1()}s.  If a third module called
@code{fun1()} the system would know which @code{fun1()} to execute based
on the types of the arguments passed to @code{fun1()}.  This ability to
determine which specific routine to execute, among several possible, is
called @emph{dynamic binding}.

At first, it may seem unclear what value polymorphism may have.  The
following example will show that polymorphism is an extremely valuable
tool, when trying to create general purpose, reusable, software
components.

Lets say you had a @emph{box} object.  That is, a data structure
representing a box (height and width), and a group of routines which
can create, modify, draw, and destroy a particular box data structure.
Now we want to create a routine to draw an arbitrary number of a
particular box.  We would probably write the routine as follows:

@example
@group
void    draw_n(box b, int n)
@{
        while (n-- > 0)
                draw(b);
@}
@end group
@end example

If you later wanted to modify @code{draw_n()} to work with circles, as
well as boxes, you would normally have to modify @code{draw_n()} as follows:

@example
@group
void    draw_n2(void *b, int n, int type)
@{
        if (type == BOX_TYPE)
                while (n-- > 0)
                        draw_box((box) b);
        else if (type == CIRCLE_TYPE)
                while (n-- > 0)
                        draw_circle((circle) b);
@}
@end group
@end example

The first problem with the @code{draw_n2()} solution is that @code{draw_n()}
had to be changed to handle the additional type.  The second problem is that
the syntax has also been changed.  This means that everywhere @code{draw_n()}
is called the syntax would have to be adjusted.  The third problem, is that
now the programmer has to keep track of which type @code{b} really is.
All this work is necessary just to add a little additional functionality.
You can imagine what would be involved in making these types of changes
to real, complex, applications.

The facility of polymorphism alleviates the need for @emph{any} of these
changes.  As an example, in Dynace, the code could have originally been
written as follows:

@example
@group

void    draw_n3(object b, int n)
@{
        while (n-- > 0)
                draw(b);
@}

@end group
@end example

If you then created a @code{circle} object, with all the appropriate supporting
functions, and passed it to @code{draw_n3()} it would work fine!  No code
changes anywhere!  You wouldn't even have to recompile @code{draw_n3()}!
You could later pass @code{draw_n3()} a @code{triangle} or a @code{rectangle}
and again, no code changes, and no re-compilation of @code{draw_n3()}.

@subsubsection Early vs. Late Binding

In the previous example of @code{draw_n3()} one should understand that
there is a @code{draw()} function defined for the @code{square}, one for the
@code{circle}, one for the @code{triangle}, and so on.  What's going on is that
the system is automatically determining which @code{draw()} to call based
on the type of @code{b}, which is kept automatically.

Some object oriented implementations have what's called @emph{early binding}.
Early binding is when the relationship between the function call, and
what's actually called is determined when the program is being compiled.
The advantage of early binding is speed.  Since the determination is being
performed at compile time, there is no runtime cost.  It is just as fast
as a normal function call.

There are, however, some serious disadvantages of early binding.  First
of all, in order for the compiler to make the determination it must know
the type of the argument at compile time.  This means that the procedure
could only work for @emph{one} type of object.  This is entirely
inadequate for the above example.  Using early binding, one would
define a function as in @code{draw_n()}.  This would allow multiple
definitions for @code{draw()}, however, it would only work for a
@code{box}.  Therefore, early binding provides only a partial solution to
the problems polymorphism can solve.

@emph{Late binding} refers to a system which determines which function
to call at runtime.  Late binding requires that the system knows, at
runtime, the type of any object.  This information is typically kept
automatically by the system, as it is in Dynace.  Late binding has the advantage
that it gives all the benefits of polymorphism.  Therefore, one could
define a very general purpose @code{draw_n3()} type of function within a system
that supports late binding.  The only disadvantage normally associated
with late binding is speed. The system must determine which
@code{draw()} function, for example, to execute at runtime.

Dynace normally uses late binding, and has a caching facility which minimizes
the runtime overhead associated with late binding.  Dynace also contains
a method for reducing the runtime overhead to virtually zero.  This
method may be used in tight loops.  Dynace can also use early binding,
however, this is discouraged.

@subsection Inheritance
Inheritance allows you to create objects which are like other objects,
or group of objects, but has some differences.  Inheritance is an
extremely powerful tool.  It enables a programmer to create a new
module by using pre-existing modules.  The only parts the programmer has
to write are the parts which make the new module unique.  In Dynace, all
this can be achieved without any access to implementation details or
source code to the pre-existing modules.

Inheritance operates in a hierarchical form.  Thus, if object @code{B}
inherits from object @code{A}, and a new object @code{C} is created
which inherits from object @code{B}, you will have the following
graphical relationship:


@example
@group
                       A
                       |
                       B
                       |
                       C
@end group
@end example

This diagram shows that object @code{C} has all the data and
functionality locally defined as part of object @code{C}, as well as all
the data and functionality of objects @code{B} and @code{A}.  Object
@code{B} has all the data and functionality locally defined as part of
object @code{B}, as well as all the data and functionality of object
@code{A}, but not object @code{C}.  Object @code{A} has only the
immediate data and functionality which are locally defined.

This idea may be extended in a multiple inheritance form pictured below:

@example
@group

               A     B
                \   /
                 \ /
                  C

@end group
@end example

In this example object @code{A} has only its locally defined data and
functionality.  Object @code{B}, also, has only its locally defined
data and functionality.  Object @code{C}, however, contains its locally
defined data and functionality, as well as the data and functionality of both
object @code{A} and @code{B}.

This idea of single and multiple inheritance may be extended arbitrarily
in both depth and width, whatever makes sense.

@page
@section Dynace
Dynace is an object oriented extension to the C language.  It implements
all of the features discussed in this manual.  Dynace is written in the
C language, and designed to be portable, efficient, powerful, and
simple to use.  Dynace was designed after an in depth study of Smalltalk--80,
the Common Lisp Object System (CLOS), Objective--C, C++, as well as
object oriented principles in general.  Dynace was designed with the following
goals:

@itemize @bullet
@item  Implement as much power and flexibility, available in languages such
as CLOS and Smalltalk--80, as possible.
@item  Create a language which was as simple to use as possible, like
Objective--C.
@item  Create a language which stuck as close as possible to the standard
C syntax.
@item  Create a language which was as efficient as possible, such as C++
and Objective--C.
@end itemize

Dynace succeeds in the satisfaction of these specific goals better than any
of the other languages above.

The remainder of this manual will deal with more specifics aspects of Dynace,
although the vast majority of these concepts are very closely related to
most of the above languages.

@subsection Object
Dynace adds to the C language only @emph{one} new data type, the @emph{object}.
An @emph{object} is a package consisting of data and the procedures
necessary to create, destroy, modify, and otherwise effect the data.  An
object always knows what type it is.  Dynace operates on objects.  An
object can be thought of as a fundamental data type.  In fact,
@code{object}, much like @code{int} or @code{long}, is a fundamental
type which may be used in declarations.  All other elements of Dynace, such
as classes, metaclasses, instances, methods, and generic functions, are
regular objects, just like any other object you may create in an
application.  In this respect, @emph{all} objects are treated exactly
the same throughout Dynace.  The exact same code which processes the pre-defined
Dynace objects, which allow Dynace to operate, process the user added objects.

Variables containing objects may be declared as follows:

@example
        object    a, b, c;
@end example

In this example three variables, @code{a, b} and @code{c}, which may
refer to objects, have been defined.

@subsection Classes & Instances
In Dynace, @emph{each} and @emph{every} object refers to an object, called
a @emph{class}, which defines that object.  One class object may define
any number of objects.  Each object which is defined by a particular
class object has the same data structure and performs the same functions
as all the other objects which are directly defined by that class
object.  The only thing which differentiates one object from another,
when they are both defined by the same class object, is the values
contained within their similar data structures.

The objects which are defined by a particular class object are called
@emph{instance} objects of that class object.  Class objects have names. 
For simplicity sake, we will call a class object with a name @code{X},
@code{X} class.  An instance object of @code{X} class is called an
instance of class @code{X}.  In general terms, we can speak directly
about classes and instances.

Now that we have some terminology under our belt, I can restate exactly
what classes and instances are, and what their relationship to each
other is.  One very important thing to keep in mind, and the main reason 
all these terminology gyrations were necessary, is that classes and instances
are both objects.  There is nothing more special about one than the other.
There is no code in Dynace which treats classes any different from instances.

A class keeps track of the structure of the data located in its instances.
A class also keeps track of the functions the instances are able to perform.
Each instance has only a single class which defines it, and it always
knows which class it is.  There may be an arbitrary number of instances of
a class, however.

@subsubsection  Superclasses
Each class may have zero or more superclasses.  Classes normally have at
least one superclass, though.  The data contained in an instance will
be the sum of the data defined directly in the instance's class, and the
data defined in all the superclasses of the instance's class.  In
addition, the instance will also contain the data defined in it's
superclass's superclasses.  This continues till there are no more
superclasses.

The instance's functionality follows the same path as it's data.

If class @code{A} is a superclass of class @code{B} then, class @code{B}
is called a @emph{subclass} of class @code{A}.

@subsubsection Instance Variables
The data contained within a particular instance are called its
@emph{instance variables}.  The values contained within an instance's
instance variables are unique to that individual instance.  Of course
every instance of one class will always have the same instance variable
structure, though.  You can think of instance variables as a C structure
definition.  Each instance of a particular class contains the same
structure.  It just contains different data within that structure.

When dealing with an instance, only the procedures which are part of the
instance's class are able to access the instance variables.  This
ensures encapsulation and data abstraction.

@subsubsection Class Variables
Each class may also contain data. This data, which is located in the class,
are called @emph{class variables}.  The data in class variables is always
unique to a particular class.  Class variables often serve as data which
is common among all the instances of the class.  With class variables
you can keep track of how many instances of the class there are, or
sums of their instance variables.  You can even keep a linked list of
a class's instances if you like.

Like instances variables, class variables may only be accessed by
specific procedures which are bound to the class.

@subsection Metaclasses
As was discussed before, all things in Dynace are @emph{objects}, and
are treated the same.  If this is true, you may be wondering why we are
discussing classes and instances like they are two different things.
The fact is, they are not!  In fact, it so happens, that like instances,
classes are also instances, instances of another class.  The classes
whose instances are themselves classes, are called @emph{metaclasses}.
And as you might imagine, metaclasses are also instances of other
classes.  This is a circular process which goes on without beginning or
end.

In this sense Dynace is quite interesting.  Every part of Dynace, and all of
its functionality, is defined by some other class.  Fortunately, though,
Dynace hides much of its complexity.  You will rarely, if ever, have to
concern yourself with any metaclasses or any aspects of its recursive
design and operation.  There is one important thing to remember, though,
every class is an instance of exactly one class (the metaclass) and has
zero or more superclasses.

@subsection Methods
As was discussed earlier, the only procedures which may access instance
variables are the procedures associated with the class of the instance.
These procedures are called @emph{methods}.  Methods are used to effect
all aspects of an instance's variables.  In fact, methods are the only
thing which can create or destroy instances.

There is one interesting thing about methods, though.  Like instance
variables, there is not, normally, a way to evoke a method.  Methods,
like instance variables, are tightly bound to the class, and there is
no easy, external access to the method.  Thus, methods, like instance
variables, are encapsulated and abstracted.

After reading about instance variables and methods being so tightly
bound and inaccessible, external to the object, one may begin to wonder
how one could ever effect an object.  Be patient.  The next few sections
will wrap up all the ideas.

These features are very important.  They are among the key elements
which allow the object oriented model to be such a powerful tool.

@subsubsection Instance Methods
@emph{Instance methods} are methods which typically operate on instance
variables.  Instance methods are conceptually located in the class of
an instance.

@subsubsection Class Methods
@emph{Class methods} are methods which typically operate on class
variables.  Class methods are conceptually located in the class of
the class, or the metaclass.

@subsection Generic Functions
Given how tightly bound and inaccessible instance (or class) variables and
methods are bound to an object, one might wonder just how anything external
to the object can cause any change in an object.  The answer to this is
@emph{generic functions}.

Generic functions act as special C functions which take one or more
arguments.  What a generic function does is check the class of the first
argument to the generic function, for a method which has been associated
to this generic function.  If one is found, it is evoked.  If one is not
found, the generic will search all the superclasses of the class of
the first argument.  This superclass search will continue until a method
is found, or there are no more superclasses to check.  If no method
is found an error is signaled.

Although this search process sounds quite involved and time consuming, it
really is not.  Dynace employs a caching scheme, so there is almost never
an actual search.

As you can see, this generic function mechanism gives Dynace quite a bit of
its power and flexibility.  Through generic functions, most of the
dynamic binding and method inheritance facilities are implemented.

Dynamic binding is achieved, for example, through the fact that class
@code{A} could have a method called @code{M1}, as can class @code{B}.
The generic function will evoke the correct method, based on the class
of its first argument.  If the first argument to the generic is an
instance of class @code{A} then class @code{A}'s @code{M1} method will
be evoked.  Likewise, if the first argument to the generic is an
instance of class @code{B} then class @code{B}'s @code{M1} method will
be evoked.  The two methods could perform entirely different functions.

The method inheritance facility of Dynace are realized by the fact that
generic functions search the superclasses until an appropriate
method is found.  It will keep searching up the tree until a class
which implements the necessary operation is found.

All the arguments passed to a generic function are passed directly to
the appropriate method, just as if the method was called directly.
Therefore, since one implementation of an @code{M1} method may take a
different number of arguments (or different types) as another @code{M1}
method, generic functions take different arguments (or different types)
depending on the method which will eventually be evoked.  In addition,
the return value of a generic is whatever the method returns.

The process of evoking a generic function may also be referred to as
sending a @emph{message} to the object which is the first argument to
the generic function.

@subsubsection Method Lookup
The order with which generic functions search for an appropriate method
is important to understand.  The first place searched is always the
class of the object in the first argument to the generic function.  If
the method is not found, the first superclass (the one listed first when
the class was created) is searched and then its superclasses.  After
checking the entire branch the method is still not found then the next
superclass to the first class will be searched.  This process will
continue until a method is found or there are no more superclasses.  If
no method is found, an error is signaled.  The search method, therefore,
is a depth-first search.  For example the following class structure:

@example
@group

               A     B
               |     |
               |     |
               C     D
                \   /
                 \ /
                  E

@end group
@end example

would be searched in the following order @code{E--C--A--D--B}.  Note that
the order of @code{C} and @code{D} is determined by the order they are
listed when class @code{E} is created.

One important point.  If the first argument to the generic function is
an instance, the class of the instance will be searched first.  And
since instance methods are kept in classes, an @emph{instance} method
will be evoked.  However, if the first argument to the generic function
is a class, the class of the class, or the @emph{metaclass}, will be
searched first.  And since class methods are kept in metaclasses, a
@emph{class} method will be evoked.

So to summarize, if the first argument to a generic function is an
instance, then an instance method will be evoked.  Likewise, if the
first argument to a generic function is a class, then a class method
will be found.

@subsection Memory Management and Garbage Collection
All objects in Dynace are dynamic entities which are allocated from the C
heap.  When a new object is created, it is allocated from the heap, but
when an object is no longer needed, it must be returned to the heap,
lest we use up all the memory.  This can be done in Dynace explicitly via
one of the disposal generics.  However, as an application gets larger
and more complex, it becomes more and more difficult to know exactly
when an object is no longer needed.  Also, there is often a larger than
desired level of complexity added to an application when trying to keep
track of when an object should be disposed of, and being sure it is.

@emph{Garbage collection} is a system used to free the programmer of
most all concern of when an object should be freed, and making sure it
is done.  The garbage collector works by marking all the objects which
your program is using, and then going through memory and freeing any
object which is not marked.  This process happens automatically, and
without any programmer intervention.  The algorithms used in Dynace assure
smooth, reliable, and efficient operation of an application without
ever having to worry about freeing any objects.

@subsection Threads
In a normal C program when one function calls another the first function
waits until the second function completes before it may continue.  The
program follows a single line, or @emph{thread}, of execution.  In a
multi-threading language a function may call a second function, and
while the second function executes the first function is able to continue
running.  Thus two threads would be running.  Either thread could even
start more threads.

Threads share the same global variables and opened files just like the
functions of a normal single threading system.  Each thread, however,
has its own stack space for local variables and function calls.

The advantage of having multiple threads is that it adds the ability to
create applications which make more efficient use of user and computer
CPU time.  While a user is updating information in an application, the
same application program could be printing a report.  Any time an
application has to perform an operation which the user would normally
have to wait for, such as report printing or other processing of
information, it could continue in a thread while the user is able to
perform other data entry/query operations in another thread.

Dynace supports multiple threads in a semi-preemptive, round robin,
priority based system.  Dynace uses the system timer to assure that
each thread gets equal access to CPU time.  Semi-preemptive operation
is achieved by the fact that Dynace performs automatic thread switching
whenever a generic function is called (if the thread's time is up).
Each thread has a priority associated with it.  Higher priority
threads get first access to CPU time, while threads of equal priority
share CPU time equally.

Dynace supports threads being put on hold, released, priority changed,
waiting for other threads, getting thread return values, and a
blocking @code{getkey} function.  Threads on hold or otherwise not
running require no CPU time.

In addition to Dynace's thread facility, Dynace also supports true
native threads provided by the underlying system.


@subsection Pipes
Once there are multiple threads running there arises a need to coordinate
and communicate among the multiple threads.  @emph{Pipes} are a
streams based facility which allows one thread to pass information
to another using a @code{puts/gets} type of interface.  @emph{Named pipes}
allows one thread to create a pipe and give it a name.  Later, if another
thread wants to use the pipe all it needs to know is the name.  It can
get access to the pipe object via the name.

Dynace supports named and anonymous pipes with blocking and non-blocking
io.

@subsection Semaphores
Thread coordination is achieved with @emph{semaphores}.  Semaphores allow
the programmer to control the number of threads which may have simultaneous
access to arbitrary system resources or blocks of code.

Semaphores may be thought of as a global variable which is initialized
to zero.  When one thread wants access to a resource it first increments
the global variable.  Later, if a different thread wants access to the
same resource it first checks the value of the global variable.  If it
is greater than zero it means that another thread is utilizing the
resource and that the second thread should wait till the first thread is
done.  When the first thread is done with the resource it decrements the
global variable.  This allows the second thread to start up and use
the resource.

Real semaphores, as opposed to global variables, automate the creation and
destruction of semaphores, the maintenance of the semaphore count, and
the holding and automatic release of threads.

Dynace supports named and anonymous counting semaphores.  Dynace semaphores
are implemented in such a fashion so that threads which are waiting for
a semaphore require no CPU time!

@subsection Dynace Kernel & Class Library
Dynace consists of two separable parts.  One part is called the @emph{kernel},
and the other is the @emph{class library}.  The kernel contains all the
raw functionality of Dynace.  It contains all the features which allow the
creation and functionality of classes, instances, methods, generics, and
garbage collection.  

The class library provides a set of general purpose classes which are
built on top of the Dynace kernel.  These classes provide basic facilities
to create object representations of all the basic C language data types.
It also provides other general use classes to handle object collection
and ordering.

The Dynace kernel is entirely functional without the class library.

