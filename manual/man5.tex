@c -*-texinfo-*-

@c  Copyright (c) 1996 Blake McBride
@c  All rights reserved.
@c
@c  Redistribution and use in source and binary forms, with or without
@c  modification, are permitted provided that the following conditions are
@c  met:
@c
@c  1. Redistributions of source code must retain the above copyright
@c  notice, this list of conditions and the following disclaimer.
@c
@c  2. Redistributions in binary form must reproduce the above copyright
@c  notice, this list of conditions and the following disclaimer in the
@c  documentation and/or other materials provided with the distribution.
@c
@c  THIS SOFTWARE IS PROVIDED BY THE COPYRIGHT HOLDERS AND CONTRIBUTORS
@c  "AS IS" AND ANY EXPRESS OR IMPLIED WARRANTIES, INCLUDING, BUT NOT
@c  LIMITED TO, THE IMPLIED WARRANTIES OF MERCHANTABILITY AND FITNESS FOR
@c  A PARTICULAR PURPOSE ARE DISCLAIMED. IN NO EVENT SHALL THE COPYRIGHT
@c  HOLDER OR CONTRIBUTORS BE LIABLE FOR ANY DIRECT, INDIRECT, INCIDENTAL,
@c  SPECIAL, EXEMPLARY, OR CONSEQUENTIAL DAMAGES (INCLUDING, BUT NOT
@c  LIMITED TO, PROCUREMENT OF SUBSTITUTE GOODS OR SERVICES; LOSS OF USE,
@c  DATA, OR PROFITS; OR BUSINESS INTERRUPTION) HOWEVER CAUSED AND ON ANY
@c  THEORY OF LIABILITY, WHETHER IN CONTRACT, STRICT LIABILITY, OR TORT
@c  (INCLUDING NEGLIGENCE OR OTHERWISE) ARISING IN ANY WAY OUT OF THE USE
@c  OF THIS SOFTWARE, EVEN IF ADVISED OF THE POSSIBILITY OF SUCH DAMAGE.

@chapter Class Library Reference
@pdfchapter{Class Library Reference}
The Dynace class library is built on top of the Dynace kernel.  It
provides a library of facilities which make the Dynace system useful to
an application programmer.  It provides various data representational
type classes which represent the normal C language data types such as
short, long, double, pointer, etc.  It also provides new data type
classes such as date.

The class library also provides generic container type classes such as
dictionaries and linked lists.  It also provides classes which implement
the ability for Dynace to run multiple threads and for those threads to
communicate and control each other.

This chapter will discuss the structure and various facilities available in
the Dynace class library.

Note that most of the Dynace kernel will run independently of the Dynace class
library.  The value of this is that should a programmer desire to create
his own class library (to replace the included class library) he may
build on top of the Dynace kernel just as the included library does.

Note that the first argument to all class methods is always the
associated class object and the first argument to all instance methods
is always an instance of the associated class.  Therefore, documentation
for the first argument of generics is not always given -- it's redundant.

There is a naming convention used with Dynace generics.  All generics
start with either a lower case ``g'' or ``v'', and are always
followed by an upper case letter.  The ones which start with ``g'' are
normal generics and may be treated like normal C functions.  The ones
which begin with ``v'' use the variable argument facilities of C and
you should, therefore, take a bit extra care when using them since
there is no compile time argument checking being done with these functions.

Since all generics start with either ``g'' or ``v'' and in order to
avoid the difficulty associated with grouping all the generics under
two letters, the first letter is dropped for indexing or heading
purposes.  Therefore if you are looking up a generic, it will always
appear in the index or header with its first letter missing.  The syntax
description and example code, however, will show the entire name.





@section Class Library Hierarchy
@pdfsection{Class Library Hierarchy}
This Dynace class library contains the following class hierarchy (which of
course includes the hierarchy described in the Kernel Reference):

@c @example
@c @group
@iftex
@break

@input classes1.tex


@end iftex
@c @end group
@c @end example


@include Array.tex
@include Association.tex
@include BitArray.tex
@include BTree.tex
@include BTreeNode.tex
@include Character.tex
@include CharacterArray.tex
@include Constant.tex
@include Date.tex
@include DateTime.tex
@include Dictionary.tex
@include DoubleFloat.tex
@include DoubleFloatArray.tex
@include File.tex
@include FindFile.tex
@include FloatArray.tex
@include IntegerArray.tex
@include IntegerAssociation.tex
@include IntegerDictionary.tex
@include Link.tex
@include LinkList.tex
@include LinkObject.tex
@include LinkObjectSequence.tex
@include LinkSequence.tex
@include LinkValue.tex
@include LongArray.tex
@include LongInteger.tex
@include LookupKey.tex
@include LowFile.tex
@include Number.tex
@include NumberArray.tex
@include ObjectArray.tex
@include ObjectAssociation.tex
@include ObjectPool.tex
@include Pipe.tex
@include Pointer.tex
@include PointerArray.tex
@include PropertyList.tex
@include Semaphore.tex
@include Sequence.tex
@include Set.tex
@include SetSequence.tex
@include ShortArray.tex
@include ShortInteger.tex
@include Socket.tex
@include Stream.tex
@include String.tex
@include StringAssociation.tex
@include StringDictionary.tex
@include Thread.tex
@include Time.tex
@include UnsignedShortArray.tex
@include UnsignedShortInteger.tex


