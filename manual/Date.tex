@page

@section Date Class
The @code{Date} class is a subclass of the @code{LongInteger}
class and used to represent dates.  The equivalent C language
representation would be a long integer in the form @code{YYYYMMDD}
so July 23, 1993 would be represented as 19930723L.





@subsection Date Class Methods
The @code{Date} class has only one class method.  The @code{New}
method is inherited from the @code{LongInteger} class.






@deffn {CalToJul} CalToJul::Date
@sp 2
@example
@group
i = gCalToJul(Date, dt);

long  i;
long  dt;
@end group
@end example
This class method returns a julian date calculated from the
input calendar date.
@example
@group
@exdent Example:

long  x;

x = gCalToJul(Date, 20000605L);  /* x = julian date 730276L */
@end group
@end example
@sp 1
See also:  @code{JulToCal::Date, Julian::Date}
@end deffn






@deffn {FixInvalidDate} FixInvalidDate::Date
@sp 2
@example
@group
i = gFixInvalidDate(Date, mode);

int  i;
int  mode;
@end group
@end example
This class method sets the method of handling invalid dates.  The
previous mode is returned.

The list of valid modes is as follows:
@example
@group
0   Does nothing.  Leaves the date invalid.
1   Changes the date to be the last day of the month.
2   Changes the date to be the corresponding day of the
    next month.
@end group
@end example

@example
@group
@exdent Example:

int  x;

x = gFixInvalidDate(Date, 1);   /*  x = the previous mode */
@end group
@end example
@sp 1
@end deffn






@deffn {JulToCal} JulToCal::Date
@sp 2
@example
@group
i = gJulToCal(Date, jdt);

long  i;
long  jdt;
@end group
@end example
This class method returns a calendar date calculated from the
input julian date.
@example
@group
@exdent Example:

long  x;

x = gCalToJul(Date, 730276L);
    /*  x = calendar date 20000605L */
@end group
@end example
@sp 1
See also:  @code{CalToJul::Date, Julian::Date}
@end deffn






@deffn {Today} Today::Date
@sp 2
@example
@group
i = gToday(Date);

object  i;
@end group
@end example
This class method creates instances of the @code{Date} class which
represents the current date contained within the system.
@example
@group
@exdent Example:

object  x;

x = gToday(Date);   /*  x = the current date  */
@end group
@end example
@sp 1
See also:  @code{FormatDate::Date, Dispose::Object}
@end deffn










@subsection Date Instance Methods
A portion of the @code{Date} class's functionality is obtained through
its inheritance of the @code{LongInteger} class.  The remaining
functionality is defined by the following specific methods.







@deffn {AddDays} AddDays::Date
@sp 2
@example
@group
i = gAddDays(i, days);

object  i;
long    days;
@end group
@end example
This method is used to add an arbitrary number of days (@code{days}) to
date @code{i}.  Leap years and the correct number of days in each month
are accounted for.  The value returned is the object passed.
@example
@group
@exdent Example:

object  dt;

dt = gNewWithLong(Date, 19940130L);
gAddDays(dt, 2L);   /*  dt contains 19940201  */
@end group
@end example
@sp 1
See also:  @code{AddMonths::Date, AddYears::Date}
@end deffn








@deffn {AddMonths} AddMonths::Date
@sp 2
@example
@group
i = gAddMonths(i, m);

object  i;
int     m;
@end group
@end example
This method is used to add an arbitrary number of months (@code{m}) to
date @code{i}.  The correct number of months in each year
is accounted for.  The value returned is the object passed.
@example
@group
@exdent Example:

object  dt;

dt = gNewWithLong(Date, 19940104L);
gAddMonths(dt, 2);   /*  dt contains 19940304  */
@end group
@end example
@sp 1
See also:  @code{AddDays::Date, AddYears::Date}
@end deffn











@deffn {AddYears} AddYears::Date
@sp 2
@example
@group
i = gAddYears(i, y);

object  i;
int     y;
@end group
@end example
This method is used to add an arbitrary number of years (@code{y}) to
date @code{i}.  The value returned is the object passed.
@example
@group
@exdent Example:

object  dt;

dt = gNewWithLong(Date, 19940104L);
gAddYears(dt, 2);   /*  dt contains 19960104  */
@end group
@end example
@sp 1
See also:  @code{AddDays::Date, AddMonths::Date}
@end deffn










@deffn {ChangeDateValue} ChangeDateValue::Date
@sp 2
@example
@group
i = gChangeDateValue(i, dt);

object  i;
long    dt;
@end group
@end example
This method is used to change the date associated with an instance of
the @code{Date} class.  Notice that this method returns the instance
being passed.  @code{dt} is the new date.
@example
@group
@exdent Example:

object  x;

x = gNewWithLong(Date, 20000101L);
gChangeLongValue(x, 20000605L);
@end group
@end example
@sp 1
See also:  @code{ChangeLongValue::Number}
@end deffn







@deffn {DateValue} DateValue::Date
@sp 2
@example
@group
dt = gDateValue(i);

object  i;
long    dt;
@end group
@end example
This method is used to obtain the @code{long} value that represents
the date associated with an instance the @code{Date} class.  Note that
this is one of the few generics which doesn't return a Dynace object.
It returns a @code{long}.
@example
@group
@exdent Example:

object  x;
long    dt;

x = gNewWithLong(LongInteger, 20000605L);
dt = gDateValue(x);    /*  dt = 20000605L  */
@end group
@end example
@sp 1
See also:  @code{LongValue::Number}
@end deffn











@deffn {DayName} DayName::Date
@sp 2
@example
@group
s = gDayName(i);

object  i;
object  s;
@end group
@end example
This method returns an instance of the @code{String} class which represents
the day of the week associated with instance @code{i}.
@example
@group
@exdent Example:

object  s, dt;

dt = gToday(Date);
s = gDayName(dt);   /*  s contains "Monday" (or whatever)  */
@end group
@end example
@sp 1
See also:  @code{FormatDate::Date, MonthName::Date}
@end deffn










@deffn {Difference} Difference::Date
@sp 2
@example
@group
r = gDifference(i, dt);

object  i;
object  dt;
long    r;
@end group
@end example
This method is used to obtain the difference, in days, between two
date objects.  Leap years and the correct number of days per month
is accounted for.
@example
@group
@exdent Example:

object  dt1, dt2;
long    r;

dt1 = gNewWithLong(Date, 19940104L);
dt2 = gNewWithLong(Date, 19940204L);
r = gDifference(dt2, dt1);    /*  r = 31L  */
@end group
@end example
@c @sp 1
@c See also:  @code{}
@end deffn







@deffn {FormatDate} FormatDate::Date
@sp 2
@example
@group
s = gFormatDate(i, fmt);

object  i;
char    *fmt;
object  s;
@end group
@end example
This method returns an instance of the @code{String} class which is a
formatted representation of the date associated with instance @code{i}.
@code{fmt} is the format specification used to determine the resulting
@code{String} object.  Each character in @code{fmt} is sequentially
processed and except for the character `%', they are simply copied to
the resulting @code{String} object.  Whenever this method encounters
the `%' character, the character following is used to determine what
aspect and format of the date represented by @code{i} should be added to
the resultant @code{String} object.  It works much like the standard C
@code{printf}.  The following table indicates the available formatting option
characters:
@example
@group
%       the % character
w       the day of the week (Friday)
M       the month name (June)
m       short month name (Feb)
d       day (8)
D       day to two places (08)
y       short year to two places (04)
Y       full year (1994)
s       day suffix (th)
n       month number (6)
N       month number to two places (06)
t       current system time (10:05 AM)
@end group
@end example
@example
@group
@exdent Example:

object  s, dt;

dt = gToday(Date);
s = gFormatDate(dt, "%w the %d%s of %M");
/*  s contains "Monday the 1st of March" (or whatever)  */
@end group
@end example
@sp 1
See also:  @code{MonthName::Date, DayName::Date}
@end deffn










@deffn {Julian} Julian::Date
@sp 2
@example
@group
jdt = gJulian(i);

object  i;
object  jdt;
@end group
@end example
This method returns a Julian date representation of the date
associated with instance @code{i}.
@example
@group
@exdent Example:

object  jdt, dt;

dt = gToday(Date);
jdt = gMonthName(dt);   /*  s contains a Julian Date  */
@end group
@end example
@sp 1
See also:  @code{CalToJul::Date, JulToCal::Date}
@end deffn








@deffn {MonthName} MonthName::Date
@sp 2
@example
@group
s = gMonthName(i);

object  i;
object  s;
@end group
@end example
This method returns an instance of the @code{String} class which represents
the month of the year associated with instance @code{i}.
@example
@group
@exdent Example:

object  s, dt;

dt = gToday(Date);
s = gMonthName(dt);   /*  s contains "June" (or whatever)  */
@end group
@end example
@sp 1
See also:  @code{FormatDate::Date, DayName::Date}
@end deffn










@deffn {StringRepValue} StringRepValue::Date
@sp 2
@example
@group
s = gStringRepValue(i);

object  i;
object  s;
@end group
@end example
This method is used to generate an instance of the @code{String} class
which represents the value associated with @code{i}.  This is often
used to print or display the value.  It is also used by
@code{PrintValue::Object} and indirectly by @code{Print::Object}
(two methods useful during the debugging phase of a project)
in order to directly print an object's value.
@example
@group
@exdent Example:

object  x;
object  s;

x = gNewWithLong(Date, 20000605L);
s = gStringRepValue(x);      /*  s represents "6/05/00"   */
@end group
@end example
@sp 1
See also:  @code{PrintValue::Object, Print::Object, FormatDate::Date}
@end deffn










@deffn {ValidDate} ValidDate::Date
@sp 2
@example
@group
r = gValidDate(i);

object  i;
int     r;
@end group
@end example
This method is used to determine the validity of a date.  If @code{i}
is a valid date 1 is returned and 0 otherwise.   Leap years and the
number of days in each month are all accounted for.
@example
@group
@exdent Example:

object  dt;
int     r;

dt = gNewWithLong(Date, 19940104L);
r = gValidDate(dt);   /*  r = 1  */
dt = gNewWithLong(Date, 19941404L);
r = gValidDate(dt);   /*  r = 0  */
dt = gNewWithLong(Date, 19940230L);
r = gValidDate(dt);   /*  r = 0  */
@end group
@end example
@c @sp 1
@c See also:  @code{}
@end deffn













