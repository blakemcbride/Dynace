@page

@section  ObjectArray Class
@pdfsection{ObjectArray Class}
This class, which is a subclass of @code{Array}, is used to represent
arbitrary shaped arrays of Dynace objects in an efficient manner.  By
using @code{ObjectArray}s it is possible to have arrays where each
element contains an arbitrary object including another arrays.
Therefore it is possible to create nested or general arrays which are
nested to arbitrary levels.


Much of the functionality of this class is implemented and documented in
the @code{Array} class.  Differences are documented in this section.



@subsection ObjectArray Class Methods
@pdfsubsection{ObjectArray Class Methods}
The only class method implemented by this class is one used to create
new @code{ObjectArray} instances.






@pdfsubsubsection {New}
@deffn {New} New::ObjectArray
@sp 2
@example
@group
ary = vNew(ObjectArray, rnk, ...)

unsigned  rnk, ...
object    ary;
@end group
@end example
This class method is used to create a new instance of @code{ObjectArray}.

@code{rnk} is the number of dimensions the new array should have.
The remaining arguments (each of type unsigned) indicates the size of
each consecutive dimension.  Note that the number of arguments following
@code{rnk} @emph{must} be the same as the value in @code{rnk}.

@code{ary} is the new array object created and will be initialized to all
@code{NULL}'s.
@example
@group
@exdent Example:

object  ary;

ary = vNew(ObjectArray, 2, 5, 4);
/*  ary is a 5x4 matrix  */
@end group
@end example
@c @sp 1
@c See also:  @code{}
@end deffn



@subsection ObjectArray Instance Methods
@pdfsubsection{ObjectArray Instance Methods}
Most instance functionality is obtained and documented in the @code{Array}
class, however, functionality which is particular to this class is documented
in this section.





@pdfsubsubsection {Value}
@deffn {Value} Value::ObjectArray
@sp 2
@example
@group
v = vValue(ary, ...);

object    ary;
unsigned  ...
object    v;
@end group
@end example
This method is used to obtain the object (or NULL) value associated with a
particular element of an instance of the @code{ObjectArray} class.

The arguments after the @code{ary} argument (each an @code{unsigned})
are used to specify the exact index into each consecutive dimension of
the array.  The number of arguments after the @code{ary} argument
@emph{must} be equal to the number of dimensions (or rank) of array
@code{ary}.  See @code{IndexOrigin::Array} for more information.
@example
@group
@exdent Example:

object  ary;
object  v;

ary = vNew(ObjectArray, 2, 5, 4);
v = vValue(ary, 1, 2);
/*  v has ary[1][2]  */
@end group
@end example
@sp 1
See also:  @code{ChangeValue::ObjectArray}
@end deffn








@pdfsubsubsection {ChangeValue}
@deffn {ChangeValue} ChangeValue::ObjectArray
@sp 2
@example
@group
ary = vChangeValue(ary, val, ...);

object    ary;
object    val;
unsigned  ...
@end group
@end example
This method is used to change the value of one element of
@code{ObjectArray} @code{ary}.

@code{val} is the value which the element of the array should be changed
to (or NULL).

The arguments after the @code{val} argument (each an @code{unsigned})
are used to specify the exact index into each consecutive dimension of
the array.  The number of arguments after the @code{val} argument
@emph{must} be equal to the number of dimensions (or rank) of array
@code{ary}.  See @code{IndexOrigin::Array} for more information.

The value returned is the modified array passed.
@example
@group
@exdent Example:

object  ary, v;

ary = vNew(ObjectArray, 2, 5, 4);
v = gNewWithInt(ShortInteger, 6);
vChangeValue(ary, v, 1, 2);
/*  ary[1][2] <- v  */
@end group
@end example
@sp 1
See also:  @code{Value::ObjectArray}
@end deffn









