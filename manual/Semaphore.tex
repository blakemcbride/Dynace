@page

@section Semaphore Class
This class provides a mechanism to coordinate the activity of multiple
threads.  It provides counting, named and unnamed semaphores.
 See the manual for a detailed description of semaphores.


See the examples included with the Dynace system for an illustration of the
use of the thread / pipe / semaphore related classes.


@subsection Semaphore Class Methods
The only two class methods associated with this class provide a
means to create new semaphore instances and to find existing
semaphores by name.








@deffn {FindStr} FindStr::Semaphore
@sp 2
@example
@group
s = gFindStr(Semaphore, name)

char    *name;
object  s;
@end group
@end example
This method is used to obtain a semaphore object from its associated name.
If found it is returned, otherwise @code{NULL} is returned.  
@c @example
@c @group
@c @exdent Example:
@c 
@c @end group
@c @end example
@sp 1
See also:  @code{New::Semaphore}
@end deffn










@deffn {New} New::Semaphore
@sp 2
@example
@group
s = gNew(Semaphore)

object  s;
@end group
@end example
This class method is used to create and initialize a new unnamed semaphore
with an initial and maximum count of 1.

The value returned (@code{s}) is the semaphore object created.
@c @example
@c @group
@c @exdent Example:
@c 
@c @end group
@c @end example
@sp 1
See also:  @code{NewSemaphore::Semaphore, FindStr::Semaphore,}
@iftex
@hfil @break @hglue .64in      
@end iftex
@code{Dispose::Semaphore}
@end deffn













@deffn {NewSemaphore} NewSemaphore::Semaphore
@sp 2
@example
@group
s = gNewSemaphore(Semaphore, name, cnt, max)

char    *name;
int     cnt, max;
object  s;
@end group
@end example
This class method is used to create and initialize a new semaphore.

@code{name} is the name of the new semaphore and may be @code{NULL} if
no name is to be associated with the semaphore.

@code{cnt} is the initial count associated with the counting semaphore
and is used to indicate the maximum number of threads which may hold
this semaphore at one time.  A typical value would be 1 and indicates
that only a single thread may hold it at a time.

@code{max} is the maximum value the count of the semaphore is allowed to
achieve.  If set higher (with @code{Release}) it will be reset to
@code{max}.  This is typically set to the same value as @code{cnt}.

The value returned (@code{s}) is the semaphore object created.  If not
kept it may be obtained later by the use of the @code{FindStr::Semaphore}
method.  
@c @example
@c @group
@c @exdent Example:
@c 
@c @end group
@c @end example
@sp 1
See also:  @code{NewSemaphore::Semaphore, FindStr::Semaphore,}
@iftex
@hfil @break @hglue .64in      
@end iftex
@code{Dispose::Semaphore}
@end deffn





@subsection Semaphore Instance Methods
The instance methods associated with this class provide a means
for locking, releasing, disposing and getting information about
semaphore instances.












@deffn {Count} Count::Semaphore
@sp 2
@example
@group
cnt = gCount(s)

object  s;
int     cnt;
@end group
@end example
This method is used to obtain the current lock count associated with
semaphore @code{s}.
@c @example
@c @group
@c @exdent Example:
@c 
@c @end group
@c @end example
@sp 1
See also:  @code{New::Semaphore, Name::Semaphore}
@end deffn









@deffn {DeepDispose} DeepDispose::Semaphore
@sp 2
@example
@group
r = gDeepDispose(s)

object  s;
object  r;     /*  NULL  */
@end group
@end example
This method is used to dispose of a semaphore.  All locks will be released
and the semaphore is disposed.  This method is the same as
@code{Dispose::Semaphore}.
@c @example
@c @group
@c @exdent Example:
@c 
@c @end group
@c @end example
@sp 1
See also:  @code{Release::Semaphore}
@end deffn












@deffn {Dispose} Dispose::Semaphore
@sp 2
@example
@group
r = gDispose(s)

object  s;
object  r;     /*  NULL  */
@end group
@end example
This method is used to dispose of a semaphore.  All locks will be released
and the semaphore is disposed.  This method is the same as
@code{DeepDispose::Semaphore}.

The value returned is always @code{NULL} and may be used to null out
the variable which contained the object being disposed in order to
avoid future accidental use.
@c @example
@c @group
@c @exdent Example:
@c 
@c @end group
@c @end example
@sp 1
See also:  @code{Release::Semaphore}
@end deffn










@deffn {Name} Name::Semaphore
@sp 2
@example
@group
nam = gName(s)

object  s;
char    *nam;
@end group
@end example
This method is used to obtain the name of semaphore @code{s}.
@c @example
@c @group
@c @exdent Example:
@c 
@c @end group
@c @end example
@sp 1
See also:  @code{New::Semaphore, Count::Semaphore}
@end deffn








@deffn {Release} Release::Semaphore
@sp 2
@example
@group
s = gRelease(s, cnt)

object  s;
int     cnt;
@end group
@end example
This method is used to release a lock on a semaphore.  The semaphore
lock count will be incremented by @code{cnt}, which is normally 1.
As soon as a semaphore's lock count goes above 0 any threads which
are waiting for the semaphore will be taken off hold and receive
the lock (limited to the number of locks available).  This method
returns the semaphore instance passed.
@c @example
@c @group
@c @exdent Example:
@c 
@c @end group
@c @end example
@sp 1
See also:  @code{WaitFor::Semaphore}
@end deffn










@deffn {WaitFor} WaitFor::Semaphore
@sp 2
@example
@group
r = gWaitFor(s)

object  s;
int     r;
@end group
@end example
This method is used to obtain a lock on semaphore @code{s}.  If the
semaphore has already exceeded its lock count the current thread
will be placed on hold until the semaphore is available.  Each call
to @code{WaitFor} uses up a single count associated with semaphore
@code{s}.  This method always returns 0.
@c @example
@c @group
@c @exdent Example:
@c 
@c @end group
@c @end example
@sp 1
See also:  @code{Release::Semaphore}
@end deffn





