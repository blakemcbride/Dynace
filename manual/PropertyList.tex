@page

@section PropertyList Class
@pdfsection{PropertyList Class}
Property lists allow arbitrary string name / Dynace object pairs to
be associated with any Dynace object.  The Dynace objects being associated
are associated and retrieved via a unique string name.  Any number of these
associations may be applied to an instance of a class with this capability.

This class is used as a mixin class.  A mixin class is one which is
sub-classed with another class (multiple inheritance) in order to give
additional functionality to the first class.  This class gives any class
which inherits from it the ability to set and retrieve arbitrary Dynace
objects which are accessed by string names (properties).

For example, if you had a class called @code{MyClass} and wanted to give it
the ability to have property lists you would create a new class called
@code{MyClassWithProperties} (for example) which would inherit from
@code{MyClass} and @code{PropertyList}.  The new class (@code{MyClassWithProperties})
would contain all the functionality of @code{MyClass} plus it would be able
to handle property lists.


@subsection PropertyList Class Methods
@pdfsubsection{PropertyList Class Methods}
There are no class methods implemented by this class.


@subsection PropertyList Instance Methods
@pdfsubsection{PropertyList Instance Methods}










@pdfsubsubsection {DisposePropertyList}
@deffn {DisposePropertyList} DisposePropertyList::PropertyList
@sp 2
@example
@group
r = gDisposePropertyList(ins);

object  ins;   /*  the instance */
object  r;     /*  ins  */
@end group
@end example
This method is used to remove all properties previously associated with an
object.  The properties will either be deep disposed or simply disassociated
depending on how they were added (see @code{PropertyPut}).  The instance
passed is returned.
@example
@group
@exdent Example:

gDisposePropertyList(ins); 
@end group
@end example
@sp 1
See also:  @code{PropertyRemove::PropertyList}
@end deffn





@pdfsubsubsection {PropertyGet}
@deffn {PropertyGet} PropertyGet::PropertyList
@sp 2
@example
@group
r = gPropertyGet(ins, pr);

object  ins;   /*  the instance */
char   *pr;    /*  the property name  */
object  r;     /*  value  */
@end group
@end example
This method is used to retrieve a property previously associated with an
object.  The property is returned otherwise, if no property with that name
was added, then @code{NULL} is returned.
@example
@group
@exdent Example:

object  val;
val = gPropertyGet(ins, "Color"); 
@end group
@end example
@sp 1
See also:  @code{PropertyPut::PropertyList, PropertyRemove::PropertyList}
@end deffn








@pdfsubsubsection {PropertyPut}
@deffn {PropertyPut} PropertyPut::PropertyList
@sp 2
@example
@group
r = gPropertyPut(ins, pr, ad, val);

object  ins;   /*  the instance */
char   *pr;    /*  the property name  */
int     ad;    /*  auto dispose flag  */
object  val;   /*  value to be associated  */
object  r;     /*  val  */
@end group
@end example
This method is used to associate an object (@code{val}) to instance
@code{ins} under the property name given by @code{pr}.  If the given
property has already been used it will be overwritten by @code{val}.

@code{ad} is a flag used to determine what happens to @code{val} once
@code{ins} is disposed or if the property is overwritten.  If @code{ad}
is 1 the object will be deep disposed otherwise (if it's zero) it will
be simply disassociated with the object and not disposed.
@example
@group
@exdent Example:

object  ins = gNew(MyClass);
gPropertyPut(ins, "Color", 1, gNewWithStr(String, "Red")); 
@end group
@end example
@sp 1
See also:  @code{PropertyGet::PropertyList, PropertyRemove::PropertyList}
@end deffn






@pdfsubsubsection {PropertyRemove}
@deffn {PropertyRemove} PropertyRemove::PropertyList
@sp 2
@example
@group
r = gPropertyRemove(ins, pr);

object  ins;   /*  the instance */
char   *pr;    /*  the property name  */
object  r;     /*  ins  */
@end group
@end example
This method is used to remove a property previously associated with an
object.  The property will either be deep disposed or simply disassociated
depending on how it was added (see @code{PropertyPut}).  The instance
passed is returned.
@example
@group
@exdent Example:

gPropertyRemove(ins, "Color"); 
@end group
@end example
@sp 1
See also:  @code{DisposePropertyList::PropertyList, PropertyRemove::PropertyList}
@end deffn






