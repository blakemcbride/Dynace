@page

@section Socket Class
The @code{Socket} class is a subclass of @code{Stream} and provides
facilities which make utilizing sockets easy and portable.  Sockets
provide a mechanism for network communication over TCP or the Internet.


At present only the client-side methods are implemented.


@subsection Socket Class Methods
The only class method is used to open a new socket which equates to
forming a connection with a server.














@deffn {SocketConnect} SocketConnect::Socket
@sp 2
@example
@group
skt = gSocketConnect(Socket, addr, port);

char    *addr;
int     port;
object  skt;
@end group
@end example
This method is used to form a connection between a client and server.
The application calling this method would be the client side.  In order
to form a connection the IP address (``123.456.789.012'') or domain name
(``prep.ai.mit.edu'') as well as the port number of the server
application must be known.  @code{addr} is a character string representing
the IP address or domain name, and @code{port} is the port number for the
server application.  This information would always be known by the organization
responsible for the server end.

@code{gSocketConnect} returns an object which represents the connection or
@code{NULL} if the connection was not made.
@example
@group
@exdent Example:

object  skt = gSocketConnect(Socket, "192.44.32.44", 4099);
@end group
@end example
@sp 1
See also:  @code{Dispose::Socket}
@end deffn










@subsection Socket Instance Methods
Instance methods associated with the @code{Socket} class enable data to
be sent to and from a client / server pair of applications.
















@deffn {Dispose} Dispose::Socket
@sp 2
@example
@group
r = gDispose(skt);

object  skt;
object  r;     /*  NULL  */
@end group
@end example
This method is used to close a socket connection and dispose of its
associated object.  There is also a @code{gDeepDispose} which performs the
same function.
@c @example
@c @group
@c @exdent Example:
@c 
@c @end group
@c @end example
@c @sp 1
@c See also:  @code{}
@end deffn















@deffn {GetErrorCode} GetErrorCode::Socket
@sp 2
@example
@group
r = gGetErrorCode(skt);

object  skt;
int     r;
@end group
@end example
This method is used to obtain an error code which describes the last
error on more detail.
@c @example
@c @group
@c @exdent Example:
@c 
@c @end group
@c @end example
@c @sp 1
@c See also:  @code{}
@end deffn












@deffn {Read} Read::Socket
@sp 2
@example
@group
r = gRead(skt, buf, n);

object   skt;
char     *buf;
unsigned n;
int      r;
@end group
@end example
This method is used to read @code{n} bytes into @code{buf} from
@code{Socket} @code{skt}.  The value returned is the actual number
of bytes read or -1 on error.  An error can occur for two reasons:
a severed connection, or a time out error.  In the case of a severed
connection one would not want to continue use of that socket since
it is probably not valid anymore.

@code{gRead} normally waits up to 60 seconds for reception of the
requested bytes.  If part of the data is received @code{gRead}
will again wait up to 60 seconds for the remaining data.  @code{gRead}
continues to wait for @emph{all} the data unless a connection error
occurs or no data is received for at least 60 seconds.
@example
@group
@exdent Example:
 
char    buf[80];

if (10 != gRead(skt, buf, 10))
        error code;
@end group
@end example
@sp 1
See also:  @code{gTimedRead::Socket}
@end deffn
















@deffn {RecvFile} RecvFile::Socket
@sp 2
@example
@group
r = gRecvFile(skt, tf);

object  skt;
char    *tf;
int     r;
@end group
@end example
This method is used to receive a whole file over a socket connection.
@code{ff} is the name (and path) of the local file to where the received
file is to be saved.

This method returns 0 on success or non-zero upon failure.  This method
makes some degree of effort (via acknowledgments and check sums) to assure
that the file was in fact correctly received.

This method is unique to Dynace which means that the server end would
have to have the correct receive function in order for this facility
to be used.
@example
@group
@exdent Example:
 
if (gRecvFile(skt, "myFile"))
      handle error;
@end group
@end example
@sp 1
See also:  @code{SendFile::Socket}
@end deffn

















@deffn {SendFile} SendFile::Socket
@sp 2
@example
@group
r = gSendFile(skt, ff, tf);

object  skt;
char    *ff;
char    *tf;
int     r;
@end group
@end example
This method is used to transmit a whole file over a socket connection.
@code{ff} is the name (and path) of the local file to be transmitted.
@code{tf} is the name of the file that the receiving end will see
(i.e. the name of the file the receiving end will think you are
sending).  If @code{tf} is @code{NULL} then @code{ff} is used.

This method returns 0 on success or non-zero upon failure.  This method
makes some degree of effort (via acknowledgments and check sums) to assure
that the file was in fact correctly received by the other end.

This method is unique to Dynace which means that the server end would
have to have the correct receive function in order for this facility
to be used.
@example
@group
@exdent Example:
 
if (gSendFile(skt, "../myFile", NULL))
      handle error;
@end group
@end example
@sp 1
See also:  @code{RecvFile::Socket}
@end deffn









@deffn {TimedRead} TimedRead::Socket
@sp 2
@example
@group
r = gTimedRead(skt, buf, n, w);

object   skt;
char     *buf;
unsigned n;
int      w;
int      r;
@end group
@end example
This method is used to read @code{n} bytes into @code{buf} from
@code{Socket} @code{skt}.  The value returned is the actual number
of bytes read or -1 on error.  An error can occur for two reasons:
a severed connection, or a time out error.  In the case of a severed
connection one would not want to continue use of that socket since
it is probably not valid anymore.

@code{gTimedRead} normally waits up to @code{w} seconds for reception of the
requested bytes.  If part of the data is received @code{gTimedRead}
will again wait up to @code{w} seconds for the remaining data.  @code{gTimedRead}
continues to wait for @emph{all} the data unless a connection error
occurs or no data is received for at least @code{w} seconds.
@example
@group
@exdent Example:
 
char    buf[80];

if (10 != gTimedRead(skt, buf, 10, 60))
        error code;
@end group
@end example
@sp 1
See also:  @code{gRead::Socket}
@end deffn



















@deffn {TimedWrite} TimedWrite::Socket
@sp 2
@example
@group
r = gTimedWrite(skt, buf, n, w);

object   skt;
char     *buf;
unsigned n;
int      w;
int      r;
@end group
@end example
This method is used to send @code{n} bytes from @code{buf} to
@code{Socket} @code{skt}.  The value returned is the actual number
of bytes sent or -1 on error.  An error can occur for two reasons:
a severed connection, or a time out error.  In the case of a severed
connection one would not want to continue use of that socket since
it is probably not valid anymore.

@code{gTimedWrite} normally waits up to @code{w} seconds for the ability to send the
requested bytes.  If part of the data is sent @code{gTimedWrite} will again
wait up to @code{w} seconds for the remaining data.  @code{gTimedWrite} continues to
wait until @emph{all} the data is sent unless a connection error occurs
or no data is able to send for at least @code{w} seconds.

Successfully sending data does not guarantee that the data was received
on the other end.  You must send back some kind of acknowledgement to
assure that.
@example
@group
@exdent Example:
 
if (12 != gTimedWrite(skt, "Hello, World", 12, 60))
        error code;
@end group
@end example
@sp 1
See also:  @code{gWrite::Socket}
@end deffn















@deffn {Write} Write::Socket
@sp 2
@example
@group
r = gWrite(skt, buf, n);

object   skt;
char     *buf;
unsigned n;
int      r;
@end group
@end example
This method is used to send @code{n} bytes from @code{buf} to
@code{Socket} @code{skt}.  The value returned is the actual number
of bytes sent or -1 on error.  An error can occur for two reasons:
a severed connection, or a time out error.  In the case of a severed
connection one would not want to continue use of that socket since
it is probably not valid anymore.

@code{gWrite} normally waits up to 60 seconds for the ability to send the
requested bytes.  If part of the data is sent @code{gWrite} will again
wait up to 60 seconds for the remaining data.  @code{gWrite} continues to
wait until @emph{all} the data is sent unless a connection error occurs
or no data is able to send for at least 60 seconds.

Successfully sending data does not guarantee that the data was received
on the other end.  You must send back some kind of acknowledgement to
assure that.
@example
@group
@exdent Example:
 
if (12 != gWrite(skt, "Hello, World", 12))
        error code;
@end group
@end example
@sp 1
See also:  @code{gTimedWrite::Socket}
@end deffn









