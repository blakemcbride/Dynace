@page

@section DateTime Class
The @code{DateTime} class uses multiple inheritance to inherit
from both the @code{Date} and the @code{Time} classes and is
used to represent precise moments in time.





@subsection DateTime Class Methods
The @code{DateTime} class has two class methods to create instances.






@deffn {NewDateTime} NewDateTime::DateTime
@sp 2
@example
@group
i = gNewDateTime(Time, dt, tm);

long    dt, tm;
object  i;
@end group
@end example
This class method creates instances of the @code{DateTime} class.  @code{dt}
represents the initial date value represented and @code{tm} represents
the initial time value.
@example
@group
@exdent Example:

object  x;

x = gNewWithLong(DateTime, 19990402L, 132345678L);
/*  x = 4/02/1999 1:23:45.678 pm  */
@end group
@end example
@sp 1
See also:  @code{Now::DateTime, Dispose::Object}
@end deffn










@deffn {Now} Now::DateTime
@sp 2
@example
@group
i = gNow(DateTime);

object  i;
@end group
@end example
This class method creates instances of the @code{DateTime} class which
represents the current time contained within the system.
@example
@group
@exdent Example:

object  x;

x = gNow(DateTime);   /*  x = the current date and time  */
@end group
@end example
@sp 1
See also:  @code{NewDateTime::DateTime, Dispose::Object}
@end deffn










@subsection DateTime Instance Methods
A portion of the @code{DateTime} class's functionality is obtained through
its inheritance of the @code{Date} class and of the @code{Time}.  The
remaining functionality is defined by the following specific methods.







@deffn {AddHours} AddHours::DateTime
@sp 2
@example
@group
i = gAddHours(i, hours);

object  i;
long    hours;
@end group
@end example
This method is used to add an arbitrary number of hours (@code{hours}) to
date and time @code{i}.  Adding an hour that goes past midnight will cause the time
to cycle around to morning and add a day.  Leap years and the correct number of
days in each month are accounted for.  The value returned is the object passed.
@example
@group
@exdent Example:

object  d;

d = gNewDateTime(DateTime, 19990402L, 234500000L);
gAddHours(d, 2L);
/*  d contains 19990403L, 14500000L
    (4/03/1999 1:45:00.000 am) */
@end group
@end example
@sp 1
See also:  @code{AddMilliseconds::DateTime, AddMinutes::DateTime,}
@iftex
@hfil @break @hglue .64in      
@end iftex
@code{AddSeconds::DateTime}
@end deffn








@deffn {AddMilliseconds} AddMilliseconds::DateTime
@sp 2
@example
@group
i = gAddMilliseconds(i, m);

object  i;
long    m;
@end group
@end example
This method is used to add an arbitrary number of milliseconds (@code{m}) to
date and time @code{i}.  Adding a millisecond that goes past midnight will cause
the time to cycle around to morning and add a day.  Leap years and the correct
number of days in each month are accounted for.  The value returned is the
object passed.
@example
@group
@exdent Example:

object  d;

d = gNewDateTime(DateTime, 19990402L, 234500000L);
gAddMilliseconds(d, 20L);
/*  d contains 19990402L, 234500000L
    (4/02/1999 11:45:00.020 pm) */
@end group
@end example
@sp 1
See also:  @code{AddHours::DateTime, AddMinutes::DateTime,}
@iftex
@hfil @break @hglue .64in      
@end iftex
@code{AddSeconds::DateTime}
@end deffn











@deffn {AddMinutes} AddMinutes::DateTime
@sp 2
@example
@group
i = gAddMinutes(i, m);

object  i;
long    m;
@end group
@end example
This method is used to add an arbitrary number of minutes (@code{m}) to
date and time @code{i}.  Adding a minute that goes past midnight will cause the time
to cycle around to morning and add a day.  Leap years and the correct number of
days in each month are accounted for.  The value returned is the object passed.
@example
@group
@exdent Example:

object  d;

d = gNewDateTime(DateTime, 19990402L, 234500000L);
gAddMinutes(d, 20L);
/*  d contains 19990403L, 500000L
    (4/03/1999 1:05:00.000 am) */
@end group
@end example
@sp 1
See also:  @code{AddHours::DateTime, AddMilliseconds::DateTime,}
@iftex
@hfil @break @hglue .64in      
@end iftex
@code{AddSeconds::DateTime}
@end deffn











@deffn {AddSeconds} AddSeconds::DateTime
@sp 2
@example
@group
i = gAddSeconds(i, s);

object  i;
long    s;
@end group
@end example
This method is used to add an arbitrary number of seconds (@code{s}) to
date and time @code{i}.  Adding a second that goes past midnight will cause
the time to cycle around to morning and add a day.  Leap years and the correct
number of days in each month are accounted for.  The value returned is the
object passed.
@example
@group
@exdent Example:

object  d;

d = gNewDateTime(DateTime, 19990402L, 234500000L);
gAddSeconds(d, 20L);
/*  d contains 19990402L, 234520000L
    (4/02/1999 11:45:20.000 pm) */
@end group
@end example
@sp 1
See also:  @code{AddHours::DateTime, AddMilliseconds::DateTime,}
@iftex
@hfil @break @hglue .64in      
@end iftex
@code{AddMinutes::DateTime}
@end deffn










@deffn {ChangeDateTimeValues} ChangeDateTimeValues::DateTime
@sp 2
@example
@group
i = gChangeDateTimeValues(i, dt, tm);

object  i;
long    dt, tm;
@end group
@end example
This method is used to change the date and time values associated
with an instance of the @code{DateTime} class.  Notice that this method
returns the instance being passed.  @code{dt} is the new date value and
@code{tm} is the new time value and.
@example
@group
@exdent Example:

object  d;

d = gNewDateTime(DateTime, 19990402L, 234500000L);
gChangeDateTimeValues(d, 19981231L, 44500000L);
/*  d contains 12/31/1998 4:45:00.000 am  */
@end group
@end example
@sp 1
See also:  @code{ChangeLongValue::DateTime, NewDateTime::DateTime}
@end deffn










@deffn {ChangeLongValue} ChangeLongValue::DateTime
@sp 2
@example
@group
i = gChangeLongValue(i, val);

object  i;
long    val;
@end group
@end example
This method is used to change the date value associated with an 
instance of the @code{DateTime} class.  Notice that this method
returns the instance being passed.  @code{val} is the new date value.
The time value associated with the instance of the @code{DateTime}
class is reset to 0L (midnight).
@example
@group
@exdent Example:

object  d;

d = gNewDateTime(DateTime, 19990402L, 234500000L);
gChangeLongValue(tm, 19991231L);
/*  d contains 12/31/1999 12:00:00.000 am  */
@end group
@end example
@sp 1
See also:  @code{ChangeDateTimeValues::DateTime, NewDateTime::DateTime}
@end deffn










@deffn {Compare} Compare::DateTime
@sp 2
@example
@group
r = gCompare(i, obj);

object  i;
object  obj;
int     r;
@end group
@end example
This method is used by the generic container classes to determine
the relationship of the values represented by @code{i} and @code{obj}. 
@code{r} is -1 if the date / time represented by @code{i} is less than
the date / time represented by @code{obj}, 1 if the date / time value of
@code{i} is greater than @code{obj}, and 0 if they are equal.
@c @example
@c @group
@c @exdent Example:
@c
@c @end group
@c @end example
@sp 1
See also:  @code{Hash::DateTime}
@end deffn







@deffn {DateTimeDifference} DateTimeDifference::DateTime
@sp 2
@example
@group
i = gDateTimeDifference(i, d, dd, td);

object  i, d;
long    *dd, *td;
@end group
@end example
This method is used to obtain the difference, in days and milliseconds,
between two time objects.  dd is loaded with the difference in days
and td is loaded with the difference in milliseconds.  Notice that this
method returns the instance being passed.  
@example
@group
@exdent Example:

object  d1, d2;
long    dd, td;

d1 = gNewDateTime(DateTime, 19990402L, 234500000L);
d2 = gNewDateTime(DateTime, 19990403L, 234500111L);
gDateTimeDifference(d2, d1, &dd, &td);
/*  dd = 1L, td = 111L  */
@end group
@end example
@c @sp 1
@c See also:  @code{}
@end deffn







@deffn {DateTimeValues} DateTimeValues::DateTime
@sp 2
@example
@group
i = gDateTimeValues(i, dt, tm);

object  i;
long    *dt, *tm;
@end group
@end example
This method is used to obtain the @code{long} values that represent
the date and time associated with an instance the @code{DateTime} class.
Notice that this method returns the instance being passed.  
@example
@group
@exdent Example:

object  x;
long    dt, tm;

x = gNewDateTime(DateTime, 19990402L, 234500000L);
gDateTimeValues(x, &dt, &tm);
/*  dt = 19990402L and tm = 234500000L  */
@end group
@end example
@sp 1
See also:  @code{NewDateTime::DateTime, ChangeDateTimeValues::Time}
@end deffn










@deffn {FormatDateTime} FormatDateTime::DateTime
@sp 2
@example
@group
s = gFormatDateTime(i, dtFmt, tmFmt);

object  i;
char    *dtFmt;
char    *tmFmt;
object  s;
@end group
@end example
This method returns an instance of the @code{String} class which is a
formatted representation of the date and time associated with instance
@code{i}.  @code{dtFmt} is the format specification (see @code{FormatDate::Date})
used to determine the date portion of the resulting @code{String} object.
@code{tmFmt} is the format specification (see @code{FormatTime::Time})
used to determine the time portion of the resulting @code{String} object.
@example
@group
@exdent Example:

object  s, dt;

dt = gNow(DateTime);
s = gFormatDateTime(dt, "%w the %d%s of %M at ", "%h:%M %p");
/*  s = "Monday the 1st of March at 9:45 am" (or whatever)  */
@end group
@end example
@sp 1
See also:  @code{FormatDate::Date, FormatTime::Time}
@end deffn








@deffn {Hash} Hash::DateTime
@sp 2
@example
@group
val = gHash(i);

object  i;
int     val;
@end group
@end example
This method is used by the generic container classes to obtain hash values
for the date / time object.  @code{val} is a hash value between 0 and a large
integer value.
@c @example
@c @group
@c @exdent Example:
@c
@c @end group
@c @end example
@sp 1
See also:  @code{Compare::DateTime}
@end deffn











@deffn {StringRepValue} StringRepValue::DateTime
@sp 2
@example
@group
s = gStringRepValue(i);

object  i;
object  s;
@end group
@end example
This method is used to generate an instance of the @code{String} class
which represents the time associated with @code{i}.  This is often
used to print or display the date and time.  It is also used by
@code{PrintValue::Object} and indirectly by @code{Print::Object}
(two methods useful during the debugging phase of a project)
in order to directly print an object's value.
@example
@group
@exdent Example:

object  x;
object  s;

x = gNewDateTime(DateTime, 19991231L, 234500000L);
s = gStringRepValue(x);
/*  s represents "1999-12-31 11:45:00.000 pm"   */
@end group
@end example
@sp 1
See also:  @code{PrintValue::Object, Print::Object}
@end deffn










@deffn {ValidDateTime} ValidDateTime::DateTime
@sp 2
@example
@group
r = gValidDateTime(i);

object  i;
int     r;
@end group
@end example
This method is used to determine the validity of a date / time.
If @code{i} is a valid date / time 1 is returned and 0 otherwise.
@example
@group
@exdent Example:

object  d;
int     r;

d = gNewDateTime(DateTime, 19991231L, 234500000L);
r = gValidDateTime(tm);   /*  r = 1  */
d = gNewDateTime(DateTime, 19941404L, 234500000L);
r = gValidDateTime(tm);   /*  r = 0  */
d = gNewDateTime(DateTime, 19991231L, 254500899L);
r = gValidDateTime(tm);   /*  r = 0  */
@end group
@end example
@c @sp 1
@c See also:  @code{}
@end deffn













