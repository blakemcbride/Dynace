@page

@section Character Class
@pdfsection{Character Class}
The @code{Character} class is used to represent the C language
@code{char} data type as a Dynace object.  It is a subclass of the
@code{Number} class.  Even though the @code{Character} class
implements most of its own functionality, it is documented as part of
the @code{Number} class because most of the interface is the same for
all subclasses of the @code{Number} class.  Differences are documented
in this section.


@subsection Character Class Methods
@pdfsubsection{Character Class Methods}
The @code{Character} class has only one class method and it is used to create
new instances of itself.



@pdfsubsubsection {NewWithChar}
@deffn {NewWithChar} NewWithChar::Character
@sp 2
@example
@group
i = gNewWithChar(Character, c);

int     c;
object  i;
@end group
@end example
This class method creates instances of the Character class.  @code{c} is
the initial value of the character being represented.  Note that
@code{c} is of type @code{int}.  That is because if you pass a character
to a generic function (which uses a variable argument declaration) the C
language will automatically promote it to an @code{int}.

The value returned is a Dynace instance object which represents the character
passed.

Note that the default disposal methods are used by this class since
there are no special storage allocation requirements.
@example
@group
@exdent Example:

object  x;

x = gNewWithChar(Character, 'a');
@end group
@end example
@sp 1
See also:  @code{CharValue::Number, ChangeCharValue::Number,}
@hfil @break @hglue .64in      @code{Dispose::Object}
@end deffn





@subsection Character Instance Methods
@pdfsubsection{Character Instance Methods}
The instance methods associated with the @code{Character} class provide
a means of changing and obtaining the value associated with an instance
of the @code{Character} class.  Most of the @code{Character} class
instance methods are documented in the @code{Number} class because of
their common interface with all other subclasses of the @code{Number}
class.  The remainder are documented in this section.






@pdfsubsubsection {FormatChar}
@deffn {FormatChar} FormatChar::Character
@sp 2
@example
@group
s = gFormatChar(i);

object  i;
object  s;
@end group
@end example
This method creates an instance of the @code{String} class which
contains the character represented by @code{i}.
@example
@group
@exdent Example:

object  x, y;

x = gNewWithChar(Character, 'a');
y = gFormatChar(x);
/*  y is a String object which represents "a"  */
@end group
@end example
@sp 1
See also:  @code{FormatNumber::Number}
@end deffn







