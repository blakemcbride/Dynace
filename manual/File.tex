@page

@section File Class
This class is used to encapsulate the standard C stream IO facility.  It
is a subclass of @code{Stream} and enables access to these routines
through an interface which is common to all subclasses of @code{Stream}.

Although this class implements most of its own functionality, it is
documented as part of the @code{Stream} class because most of the
interface is the same for all subclasses of @code{Stream}.  Differences
are documented in this section.


@subsection File Class Methods
The class methods associated with this class are methods used to
open or create files.




@deffn {Flush} Flush::File
@sp 2
@example
@group
i = gFlush(File);

int  i;
@end group
@end example
This class method is used to flush all files opened with the
stream IO system to disk.  It returns the number of opened
streams.
@example
@group
@exdent Example:

gFlush(File);
@end group
@end example
@sp 1
See also:  @code{instance Flush::File}
@end deffn







@deffn {OpenFile} OpenFile::File
@sp 2
@example
@group
i = gOpenFile(File, name, mode);

char    *name;  /*  file name  */
char    *mode;  /*  file mode  */
object  i;
@end group
@end example
This class method is used to open or create a normal file using the
C stream IO facility.  The @code{name} and @code{mode} parameters
correspond to the local C library function @code{fopen}.  The
local library manuals can more completely describe those arguments.
The value returned is an object which refers to the opened file.
If the file cannot be opened a @code{NULL} will be returned.
@example
@group
@exdent Example:

object  f;

f = gOpenFile(File, "myfile", "r");
@end group
@end example
@sp 1
See also:  @code{OpenTempFile::File, Dispose::File, Read::Stream}
@end deffn





@deffn {OpenTempFile} OpenTempFile::File
@sp 2
@example
@group
i = gOpenTempFile(File);

object  i;
@end group
@end example
This class method is used to create a tempoary file using the C stream
IO facility.  The location of the file will be dictated by the
@code{TMP} or @code{TEMP} environment variables (in that order).  If
those variables are not defined then the current directory is used.  The
file is created with the @code{"w+"} binary mode.

The value returned is an object which refers to the opened file.  Since
this is a temporary file, disposing of this object will also cause the
file to be removed.  If the file cannot be opened a @code{NULL} will be
returned.
@example
@group
@exdent Example:

object  f;

f = gOpenTempFile(File);
@end group
@end example
@sp 1
See also:  @code{SetTempSubDir::File, OpenFile::File, Dispose::File, Read::Stream}
@end deffn







@deffn {SetTempSubDir} SetTempSubDir::File
@sp 2
@example
@group
i = gSetTempSubDir(File, dir);

char  *dir;
object  i;
@end group
@end example
This class method is used to specify a path for the system to use to
create temporary files (using @code{gOpenTempFile}.  The given directory
is appended to the normal system temp path.  For example, if the system
path is `@code{/tmp}' and @code{dir} is passed @code{"App"} then the
path used will be `@code{/tmp/App}'.

The value returned is @code{File}.
@example
@group
@exdent Example:

gSetTempSubDir(File, "Application");
@end group
@end example
@sp 1
See also:  @code{OpenTempFile::File}
@end deffn





@subsection File Instance Methods
The instance methods associated with this class are used to read and write
data to the file represented by the instance.

Although this class implements most of its own functionality, it is
documented as part of the @code{Stream} class because most of the
interface is the same for all subclasses of @code{Stream}.  Differences
are documented in this section.





@deffn {DeepDispose} DeepDispose::File
@sp 2
@example
@group
r = gDeepDispose(i);

object  i;
object  r;     /*  NULL  */
@end group
@end example
This method is used to close the file associated with the instance and
dispose of the instance.  It performs the same function as
@code{Dispose::File}.  

The value returned is always @code{NULL} and may be used to null out
the variable which contained the object being disposed in order to
avoid future accidental use.
@c @example
@c @group
@c @exdent Example:
@c 
@c @end group
@c @end example
@sp 1
See also:  @code{Dispose::File, OpenFile::File}
@end deffn











@deffn {Dispose} Dispose::File
@sp 2
@example
@group
r = gDispose(i);

object  i;
object  r;     /*  NULL  */
@end group
@end example
This method is used to close the file associated with the instance and
dispose of the instance.  It performs the same function as
@code{DeepDispose::File}.  

The value returned is always @code{NULL} and may be used to null out
the variable which contained the object being disposed in order to
avoid future accidental use.
@c @example
@c @group
@c @exdent Example:
@c 
@c @end group
@c @end example
@sp 1
See also:  @code{DeepDispose::File, OpenFile::File}
@end deffn






@deffn {Flush} Flush::File
@sp 2
@example
@group
i = gFlush(file);

object file;
int  i;
@end group
@end example
This method is used to flush the stream associated with
file to disk.  It returns zero on success.
@example
@group
@exdent Example:

gFlush(file);
@end group
@end example
@sp 1
See also:  @code{class Flush::File}
@end deffn










@deffn {Name} Name::File
@sp 2
@example
@group
n = gName(i);

object  i;
char    *n;  /*  the file name  */
@end group
@end example
This method is used to obtain the name of the file represented by @code{i}.
@c @example
@c @group
@c @exdent Example:
@c 
@c @end group
@c @end example
@c @sp 1
@c See also:  @code{}
@end deffn







@deffn {PointerValue} PointerValue::File
@sp 2
@example
@group
fp = gPointerValue(i);

object  i;
FILE    *fp;
@end group
@end example
This method is used to obtain the file pointer associated with @code{i}.
@c @example
@c @group
@c @exdent Example:
@c 
@c @end group
@c @end example
@c @sp 1
@c See also:  @code{}
@end deffn





