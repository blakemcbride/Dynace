@page

@section Time Class
The @code{Time} class uses the same set of instance and class
methods as the @code{LongInteger} class.  The equivalent C language
representation would be a long integer in the form @code{HHMMSSLLL}
so 1:23:45.678 pm would be represented as 132345678L.





@subsection Time Class Methods
The @code{Time} class has two class methods to create instances.






@deffn {NewWithLong} NewWithLong::Time
@sp 2
@example
@group
i = gNewWithLong(Time, tm);

long    tm;
object  i;
@end group
@end example
This class method creates instances of the @code{Time} class.  @code{tm}
represents the initial time value represented.
@example
@group
@exdent Example:

object  x;

x = gNewWithLong(Time, 132345678L);  /* x = 1:23:45.678 pm  */
@end group
@end example
@sp 1
See also:  @code{Now::Time, FormatTime::Time, Dispose::Object}
@end deffn










@deffn {Now} Now::Time
@sp 2
@example
@group
i = gNow(Time);

object  i;
@end group
@end example
This class method creates instances of the @code{Time} class which
represents the current time contained within the system.
@example
@group
@exdent Example:

object  x;

x = gNow(Time);   /*  x = the current time  */
@end group
@end example
@sp 1
See also:  @code{NewWithLong::Time, FormatTime::Time, Dispose::Object}
@end deffn










@subsection Time Instance Methods
A portion of the @code{Time} class's functionality is identical to
the functionality of the @code{LongInteger} class.  The remaining
functionality is defined by the following specific methods.







@deffn {AddHours} AddHours::Time
@sp 2
@example
@group
i = gAddHours(i, hours);

object  i;
long    hours;
@end group
@end example
This method is used to add an arbitrary number of hours (@code{hours}) to
time @code{i}.  Adding an hour that goes past midnight will cause the time
to cycle around to morning.  The value returned is the object passed.
@example
@group
@exdent Example:

object  tm;

tm = gNewWithLong(Time, 234500000L);
gAddHours(tm, 2L);  /* tm = 14500000L (1:45:00.000 am) */
@end group
@end example
@sp 1
See also:  @code{AddMilliseconds::Time, AddMinutes::Time, AddSeconds::Time}
@end deffn








@deffn {AddMilliseconds} AddMilliseconds::Time
@sp 2
@example
@group
i = gAddMilliseconds(i, m);

object  i;
long    m;
@end group
@end example
This method is used to add an arbitrary number of milliseconds (@code{m}) to
time @code{i}.  Adding a millisecond that goes past midnight will cause the time
to cycle around to morning.  The value returned is the object passed.
@example
@group
@exdent Example:

object  tm;

tm = gNewWithLong(Time, 234500000L);
gAddMilliseconds(tm, 20L);
    /* tm = 500020L (11:45:00.020 pm) */
@end group
@end example
@sp 1
See also:  @code{AddHours::Time, AddMinutes::Time, AddSeconds::Time}
@end deffn











@deffn {AddMinutes} AddMinutes::Time
@sp 2
@example
@group
i = gAddMinutes(i, m);

object  i;
long    m;
@end group
@end example
This method is used to add an arbitrary number of minutes (@code{m}) to
time @code{i}.  Adding a minute that goes past midnight will cause the time
to cycle around to morning.  The value returned is the object passed.
@example
@group
@exdent Example:

object  tm;

tm = gNewWithLong(Time, 234500000L);
gAddMinutes(tm, 20L);  /* tm = 500000L (1:05:00.000 am) */
@end group
@end example
@sp 1
See also:  @code{AddHours::Time, AddMilliseconds::Time, AddSeconds::Time}
@end deffn











@deffn {AddSeconds} AddSeconds::Time
@sp 2
@example
@group
i = gAddSeconds(i, s);

object  i;
long    s;
@end group
@end example
This method is used to add an arbitrary number of seconds (@code{s}) to
time @code{i}.  Adding a second that goes past midnight will cause the time
to cycle around to morning.  The value returned is the object passed.
@example
@group
@exdent Example:

object  dt;

tm = gNewWithLong(Time, 234500000L);
gAddSeconds(tm, 20L);  /* tm = 520000L (11:45:20.000 pm) */
@end group
@end example
@sp 1
See also:  @code{AddHours::Time, AddMilliseconds::Time, AddMinutes::Time}
@end deffn










@deffn {ChangeLongValue} ChangeLongValue::Time
@sp 2
@example
@group
i = gChangeLongValue(i, val);

object  i;
long    val;
@end group
@end example
This method is used to change the time value associated with an 
instance of the @code{Time} class.  Notice that this method
returns the instance being passed.  @code{val} is the new time value.
@example
@group
@exdent Example:

object  tm;

tm = gNewWithLong(Time, 234500000L);
gChangeLongValue(tm, 44500000L);   /*  tm = 4:45:00.000 am  */
@end group
@end example
@sp 1
See also:  @code{ChangeTimeValue::Time, NewWithLong::Time}
@end deffn










@deffn {ChangeTimeValue} ChangeTimeValue::Time
@sp 2
@example
@group
i = gChangeTimeValue(i, val);

object  i;
long    val;
@end group
@end example
This method is used to change the time value associated with an 
instance of the @code{Time} class.  Notice that this method
returns the instance being passed.  @code{val} is the new time value.
@example
@group
@exdent Example:

object  tm;

tm = gNewWithLong(Time, 234500000L);
gChangeTimeValue(tm, 44500000L);   /*  tm = 4:45:00.000 am  */
@end group
@end example
@sp 1
See also:  @code{ChangeLongValue::Time, NewWithLong::Time}
@end deffn










@deffn {ChangeValue} ChangeValue::Time
@sp 2
@example
@group
i = gChangeValue(i, v);

object  i;
object  v;
@end group
@end example
This method is used to change the value associated with an instance of
the @code{Time} class.  Notice that this method returns the instance
being passed.  @code{v} is an object which represents the new time value.
@example
@group
@exdent Example:

object  x, y;

x = gNewWithLong(Time, 234500000L);
y = gNewWithLong(Time, 44500000L);
gChangeValue(x, y);    /*  x is changed to 4:45:00.000 am  */
@end group
@end example
@sp 1
See also:  @code{ChangeLongValue::Time, ChangeTimeValue::Time}
@end deffn










@deffn {Compare} Compare::Time
@sp 2
@example
@group
r = gCompare(i, obj);

object  i;
object  obj;
int     r;
@end group
@end example
This method is used by the generic container classes to determine
the relationship of the values represented by @code{i} and @code{obj}. 
@code{r} is -1 if the time represented by @code{i} is less than
the time represented by @code{obj}, 1 if the time value of @code{i}
is greater than @code{obj}, and 0 if they are equal.
@c @example
@c @group
@c @exdent Example:
@c
@c @end group
@c @end example
@sp 1
See also:  @code{Hash::Time}
@end deffn







@deffn {Difference} Difference::Time
@sp 2
@example
@group
r = gDifference(i, tm);

object  i;
object  tm;
long    r;
@end group
@end example
This method is used to obtain the difference, in milliseconds,
between two time objects.  
@example
@group
@exdent Example:

object  tm1, tm2;
long    r;

tm1 = gNewWithLong(Time, 234515023L);
tm2 = gNewWithLong(Time, 234515000L);
r = gDifference(dt2, dt1);    /*  r = 23L  */
@end group
@end example
@c @sp 1
See also:  @code{TimeDifference::Time}
@end deffn







@deffn {FormatTime} FormatTime::Time
@sp 2
@example
@group
s = gFormatTime(i, fmt);

object  i;
char    *fmt;
object  s;
@end group
@end example
This method returns an instance of the @code{String} class which is a
formatted representation of the time associated with instance @code{i}.
@code{fmt} is the format specification used to determine the resulting
@code{String} object.  Each character in @code{fmt} is sequentially
processed and except for the character `%', they are simply copied to
the resulting @code{String} object.  Whenever this method encounters
the `%' character, the character following is used to determine what
aspect and format of the time represented by @code{i} should be added to
the resultant @code{String} object.  It works much like the standard C
@code{printf}.  The following table indicates the available formatting option
characters:
@example
@group
%       the % character
g       hour in military (government) time.
        Eleven o'clock is 23:00 (5)
G       the hour to two places in military (government)
        time (05)
h       hour (5)
H       the hour to two places (05)
m       minutes (5)
M       the minutes to two places (05)
s       seconds (5)
S       the seconds to two places (05)
l       milliseconds (20)
L       the milliseconds to three places (020)
p       the am / pm flag in lower case (pm)
p       the am / pm flag in upper case (PM)
@end group
@end example
@example
@group
@exdent Example:

object  s, tm;

tm = gNow(Time);
s = gFormatTime(tm, "%h:%M:%S.%L %p");
/*  s contains "5:45:23.678 pm" (or whatever)  */
@end group
@end example
@sp 1
See also:  @code{StringRepValue::Time}
@end deffn










@deffn {Hash} Hash::Time
@sp 2
@example
@group
val = gHash(i);

object  i;
int     val;
@end group
@end example
This method is used by the generic container classes to obtain hash values
for the time object.  @code{val} is a hash value between 0 and a large
integer value.
@c @example
@c @group
@c @exdent Example:
@c
@c @end group
@c @end example
@sp 1
See also:  @code{Compare::Time}
@end deffn











@deffn {Hours} Hours::Time
@sp 2
@example
@group
h = gHours(i);

object  i;
int     h;
@end group
@end example
This method returns the number of hours associated with an
instance of the @code{Time} class.
@example
@group
@exdent Example:

object  tm;

tm = gToday(Time);
h = gHours(tm);   /*  h contains 14 (or whatever)  */
@end group
@end example
@sp 1
See also:  @code{Milliseconds::Time, Minutes::Time, Seconds::Time}
@end deffn
















@deffn {LongValue} LongValue::Time
@sp 2
@example
@group
val = gLongValue(i);

object  i;
long    val;
@end group
@end example
This method is used to obtain the @code{long} value that represents
the time associated with an instance the @code{Time} class.  Note that
this is one of the few generics which doesn't return a Dynace object.
It returns a @code{long}.
@example
@group
@exdent Example:

object  x;
long    val;

x = gNewWithLong(LongInteger, 234500000L);
val = gLongValue(x);    /*  val = 234500000L  */
@end group
@end example
@sp 1
See also:  @code{NewWithLong::Time, ChangeLongValue::Time, TimeValue::Time}
@end deffn










@deffn {Milliseconds} Milliseconds::Time
@sp 2
@example
@group
m = gMilliseconds(i);

object  i;
int     m;
@end group
@end example
This method returns the number of milliseconds associated with an
instance of the @code{Time} class.
@example
@group
@exdent Example:

object  tm;

tm = gToday(Time);
m = gMilliseconds(tm);   /*  m contains 154 (or whatever)  */
@end group
@end example
@sp 1
See also:  @code{Hours::Time, Minutes::Time, Seconds::Time}
@end deffn
















@deffn {Minutes} Minutes::Time
@sp 2
@example
@group
m = gMinutes(i);

object  i;
int     m;
@end group
@end example
This method returns the number of minutes associated with an
instance of the @code{Time} class.
@example
@group
@exdent Example:

object  tm;

tm = gToday(Time);
m = gMinutes(tm);   /*  h contains 55 (or whatever)  */
@end group
@end example
@sp 1
See also:  @code{Hours::Time, Milliseconds::Time, Seconds::Time}
@end deffn
















@deffn {Seconds} Seconds::Time
@sp 2
@example
@group
m = gSeconds(i);

object  i;
int     s;
@end group
@end example
This method returns the number of seconds associated with an
instance of the @code{Time} class.
@example
@group
@exdent Example:

object  tm;

tm = gToday(Time);
s = gSeconds(tm);   /*  s contains 30 (or whatever)  */
@end group
@end example
@sp 1
See also:  @code{Hours::Time, Milliseconds::Time, Minutes::Time}
@end deffn
















@deffn {StringRepValue} StringRepValue::Time
@sp 2
@example
@group
s = gStringRepValue(i);

object  i;
object  s;
@end group
@end example
This method is used to generate an instance of the @code{String} class
which represents the time associated with @code{i}.  This is often
used to print or display the time.  It is also used by
@code{PrintValue::Object} and indirectly by @code{Print::Object}
(two methods useful during the debugging phase of a project)
in order to directly print an object's value.
@example
@group
@exdent Example:

object  x;
object  s;

x = gNewWithLong(Time, 234500000L);
s = gStringRepValue(x);
   /*  s represents "11:45:00.000 pm"  */
@end group
@end example
@sp 1
See also:  @code{PrintValue::Object, Print::Object, FormatTime::Time,}
@iftex
@hfil @break @hglue .64in      
@end iftex
@code{TimeStringRepValue::Time}
@end deffn










@deffn {TimeDifference} TimeDifference::Time
@sp 2
@example
@group
r = gTmieDifference(i, tm);

object  i;
object  tm;
long    r;
@end group
@end example
This method is used to obtain the difference, in milliseconds,
between two time objects.  
@example
@group
@exdent Example:

object  tm1, tm2;
long    r;

tm1 = gNewWithLong(Time, 234515023L);
tm2 = gNewWithLong(Time, 234515000L);
r = gTimeDifference(dt2, dt1);    /*  r = 23L  */
@end group
@end example
@c @sp 1
See also:  @code{Difference::Time}
@end deffn







@deffn {TimeStringRepValue} TimeStringRepValue::Time
@sp 2
@example
@group
s = gTimeStringRepValue(i);

object  i;
object  s;
@end group
@end example
This method is used to generate an instance of the @code{String} class
which represents the time associated with @code{i}.  This is often
used to print or display the time.
@example
@group
@exdent Example:

object  x;
object  s;

x = gNewWithLong(Time, 234500000L);
s = gTimeStringRepValue(x);
    /* s represents "11:45:00.000 pm"  */
@end group
@end example
@sp 1
See also:  @code{PrintValue::Object, Print::Object, FormatTime::Time,}
@iftex
@hfil @break @hglue .64in      
@end iftex
@code{StringRepValue::Time}
@end deffn










@deffn {TimeValue} TimeValue::Time
@sp 2
@example
@group
val = gTimeValue(i);

object  i;
long    val;
@end group
@end example
This method is used to obtain the @code{long} value that represents
the time associated with an instance the @code{Time} class.  Note that
this is one of the few generics which doesn't return a Dynace object.
It returns a @code{long}.
@example
@group
@exdent Example:

object  x;
long    val;

x = gNewWithLong(LongInteger, 234500000L);
val = gTimeValue(x);    /*  val = 234500000L  */
@end group
@end example
@sp 1
See also:  @code{NewWithLong::Time, ChangeLongValue::Time, LongValue::Time}
@end deffn










@deffn {ValidTime} ValidTime::Time
@sp 2
@example
@group
r = gValidTime(i);

object  i;
int     r;
@end group
@end example
This method is used to determine the validity of a time.  If @code{i}
is a valid time 1 is returned and 0 otherwise.
@example
@group
@exdent Example:

object  tm;
int     r;

tm = gNewWithLong(Time, 234515899L);
r = gValidTime(tm);   /*  r = 1  */
tm = gNewWithLong(Time, 254500899L);
r = gValidTime(tm);   /*  r = 0  */
tm = gNewWithLong(Time, 114567000L);
r = gValidTime(tm);   /*  r = 0  */
@end group
@end example
@c @sp 1
@c See also:  @code{}
@end deffn













