@page

@section Number Class
@pdfsection{Number Class}
The @code{Number} class is an abstract class used to combine common
functionality of the primitive C numeric data types.  Although most of the
functionality is actually implemented in the subclasses of the @code{Number}
class, they are documented here because of their common interface.


@subsection Number Class Methods
@pdfsubsection{Number Class Methods}
The @code{Number} class has no specific class methods.


@subsection Number Instance Methods
@pdfsubsection{Number Instance Methods}
Although most of the instance methods are actually implemented in the
subclasses of the @code{Number} class, they are documented here because
of their common interface.  Regardless of the numeric type being
represented there has been every effort made to make a standard set of
interfaces to these data elements.  Therefore, you are able to change
the value associated with an instance of the @code{ShortInteger} class
to a @code{double} value of 3.141, however, when you retrieve the
@code{double} value you will only get back 3.0.  (Of course you
wouldn't loose the fractional part if you were dealing with an instance
of the @code{DoubleFloat} class.)





@pdfsubsubsection {ChangeCharValue}
@deffn {ChangeCharValue} ChangeCharValue::Number
@sp 2
@example
@group
i = gChangeCharValue(i, c);

object  i;
int     c;
@end group
@end example
This method is used to change the value associated with an instance of
a subclass of the @code{Number} class.  Notice that this method
returns the instance being passed.  @code{c} is the new value.
If it is not the same type as @code{i} it will be converted.

Note that @code{c} is of type @code{int}.  That is because if you pass a
character to a generic function (which uses a variable argument
declaration) the C language will automatically promote it to an
@code{int}.
@example
@group
@exdent Example:

object  x;

x = gNewWithChar(Character, 'a');
gChangeCharValue(x, 'b');
@end group
@end example
@sp 1
See also:  @code{CharValue::Number}
@end deffn












@pdfsubsubsection {ChangeDoubleValue}
@deffn {ChangeDoubleValue} ChangeDoubleValue::Number
@sp 2
@example
@group
i = gChangeDoubleValue(i, val);

object  i;
double  val;
@end group
@end example
This method is used to change the value associated with an instance of
a subclass of the @code{Number} class.  Notice that this method
returns the instance being passed.  @code{val} is the new value.
If it is not the same type as @code{i} it will be converted.
@example
@group
@exdent Example:

object  x;

x = gNewWithDouble(DoubleFloat, 3.14);
gChangeDoubleValue(x, 7.56);
@end group
@end example
@sp 1
See also:  @code{NewWithDouble::DoubleFloat, DoubleValue::Number}
@end deffn














@pdfsubsubsection {ChangeLongValue}
@deffn {ChangeLongValue} ChangeLongValue::Number
@sp 2
@example
@group
i = gChangeLongValue(i, val);

object  i;
long    val;
@end group
@end example
This method is used to change the value associated with an instance of
a subclass of the @code{Number} class.  Notice that this method
returns the instance being passed.  @code{val} is the new value.
If it is not the same type as @code{i} it will be converted.
@example
@group
@exdent Example:

object  x;

x = gNewWithLong(LongInteger, 55L);
gChangeLongValue(x, 66L);
@end group
@end example
@sp 1
See also:  @code{NewWithLong::LongInteger, LongValue::Number}
@end deffn










@pdfsubsubsection {ChangeShortValue}
@deffn {ChangeShortValue} ChangeShortValue::Number
@sp 2
@example
@group
i = gChangeShortValue(i, s);

object  i;
int     s;
@end group
@end example
This method is used to change the value associated with an instance of
a subclass of the @code{Number} class.  Notice that this method
returns the instance being passed.  @code{s} is the new value.
If it is not the same type as @code{i} it will be converted.

Note that @code{s} is of type @code{int}.  That is because if you pass a
@code{short} to a generic function (which uses a variable argument
declaration) the C language will automatically promote it to an
@code{int}.
@example
@group
@exdent Example:

object  x;

x = gNewWithInt(ShortInteger, 55);
gChangeShortValue(x, 66);
@end group
@end example
@sp 1
See also:  @code{NewWithInt::ShortInteger, ShortValue::Number}
@end deffn









@pdfsubsubsection {ChangeUShortValue}
@deffn {ChangeUShortValue} ChangeUShortValue::Number
@sp 2
@example
@group
i = gChangeUShortValue(i, s);

object  i;
unsigned  s;
@end group
@end example
This method is used to change the value associated with an instance of
a subclass of the @code{Number} class.  Notice that this method
returns the instance being passed.  @code{s} is the new value.
If it is not the same type as @code{i} it will be converted.

Note that @code{s} is of type @code{unsigned}.  That is because if you pass a
@code{short} to a generic function (which uses a variable argument
declaration) the C language will automatically promote it to an
@code{unsigned}.
@example
@group
@exdent Example:

object  x;

x = gNewWithUnsigned(UnsignedShortInteger, 55);
gChangeUShortValue(x, 66);
@end group
@end example
@sp 1
See also:  @code{NewWithUnsigned::UnsignedShortInteger}
@end deffn










@pdfsubsubsection {ChangeValue}
@deffn {ChangeValue} ChangeValue::Number
@sp 2
@example
@group
i = gChangeValue(i, v);

object  i;
object  v;
@end group
@end example
This method is used to change the value associated with an instance of a
subclass of the @code{Number} class.  Notice that this method returns
the instance being passed.  @code{v} is an object which represents the
new value.  If it is not the same type as @code{i} it will be converted.
Note that this conversion does not effect the object @code{v}, just
the value changed in @code{i}.
@example
@group
@exdent Example:

object  x, y;

x = gNewWithInt(ShortInteger, 88);
y = gNewWithDouble(DoubleFloat, 3.141);
gChangeValue(x, y);    /*  x is changed to 3  */
@end group
@end example
@sp 1
See also:  @code{ChangeCharValue::Number, ChangeShortValue::Number, etc.}
@end deffn










@pdfsubsubsection {CharValue}
@deffn {CharValue} CharValue::Number
@sp 2
@example
@group
c = gCharValue(i);

object  i;
char    c;
@end group
@end example
This method is used to obtain the @code{char} value associated with an
instance of a subclass of the @code{Number} class.  Note that this is
one of the few generics which doesn't return a Dynace object.  It
returns a @code{char}.  If the instance does not represent an instance
of the @code{Character} class, the returned value of whatever is
represented will be converted to a @code{char}.
@example
@group
@exdent Example:

object  x;
char    c;

x = gNewWithChar(Character, 'a');
c = gCharValue(x);
@end group
@end example
@sp 1
See also:  @code{NewWithChar::Character, ChangeValue::Number}
@end deffn








@pdfsubsubsection {Compare}
@deffn {Compare} Compare::Number
@sp 2
@example
@group
r = gCompare(i, obj);

object  i;
object  obj;
int     r;
@end group
@end example
This method is used by the generic container classes to determine
the relationship of the values represented by @code{i} and @code{obj}. 
@code{r} is -1 if the value represented by @code{i} is less than
the value represented by @code{obj}, 1 if the value of @code{i}
is greater than @code{obj}, and 0 if they are equal.  @code{obj}
may be any type of object.
@c @example
@c @group
@c @exdent Example:
@c
@c @end group
@c @end example
@sp 1
See also:  @code{Hash::Number}
@end deffn







@pdfsubsubsection {DoubleValue}
@deffn {DoubleValue} DoubleValue::Number
@sp 2
@example
@group
val = gDoubleValue(i);

object  i;
double  val;
@end group
@end example
This method is used to obtain the @code{double} value associated
with an instance of a subclass of the @code{Number} class.  Note that
this is one of the few generics which doesn't return a Dynace object.
It returns a @code{double}.  If the instance does not represent
an instance of the @code{DoubleFloat} class, the returned value
of whatever is represented will be converted to a @code{double}.
@example
@group
@exdent Example:

object  x;
double  val;

x = gNewWithDouble(DoubleFloat, 7.62);
val = gDoubleValue(x);
@end group
@end example
@sp 1
See also:  @code{NewWithDouble::DoubleFloat, ChangeValue::Number}
@end deffn










@pdfsubsubsection {FormatNumber}
@deffn {FormatNumber} FormatNumber::Number
@sp 2
@example
@group
val = gFormatNumber(i, msk, wth, dp);

object  i;      /*  object to be formatted      */
char    *msk;   /*  format mask                 */
int     wth;    /*  field width                 */
int     dp;     /*  number of decimal places    */
object  val;    /*  String value                */
@end group
@end example
This method is used to convert a number into a nicely formatted string.
@code{i} is the numeric object to be formatted.

@code{msk} is a character string which tells the formatter specific
which formatting options to use.  Each character in the string
represents a different option.  The order is not significant.  The
following table lists the available options:
@example
@group
B   blank if zero
C   add commas every 3 digits to the left of the decimal point
L   left justify
P   put parentheses around negative numbers
Z   zero fill
D   floating dollar sign
U   upper-case letters in conversion (used in bases greater
    than 10)
R   add a percent sign to the end of the number
@end group
@end example
@code{wth} indicates the width of the field and hence the length of the
resultant @code{String} object.  If @code{wth} <= 0 @code{FormatNumber} will
calculate and use the minimum width which will represent @code{i}.

@code{dp} indicates the number of decimal places that should appear in
the string.  If @code{dp} is less than the number of decimal places in
@code{i} rounding will occur.  However if @code{dp} < 0 Nfmt will
calculate and use the minimum number of decimal places required to fully
represent @code{i}.

The value returned by this method is an instance of the @code{String} class.
@example
@group
@exdent Example:

object  x, y;

x = gNewWithDouble(DoubleFloat, 1234567.127);
y = gFormatNumber(x, "C", 0, 2);    /*  y = "1,234,567.13"  */
@end group
@end example
@sp 1
See also:  @code{StringRepValue::Number}
@end deffn











@pdfsubsubsection {Hash}
@deffn {Hash} Hash::Number
@sp 2
@example
@group
val = gHash(i);

object  i;
int     val;
@end group
@end example
This method is used by the generic container classes to obtain hash values
for the object.  @code{val} is a hash value between 0 and a large integer
value.
@c @example
@c @group
@c @exdent Example:
@c
@c @end group
@c @end example
@sp 1
See also:  @code{Compare::Number}
@end deffn











@pdfsubsubsection {LongValue}
@deffn {LongValue} LongValue::Number
@sp 2
@example
@group
val = gLongValue(i);

object  i;
long    val;
@end group
@end example
This method is used to obtain the @code{long} value associated
with an instance of a subclass of the @code{Number} class.  Note that
this is one of the few generics which doesn't return a Dynace object.
It returns a @code{long}.  If the instance does not represent
an instance of the @code{LongInteger} class, the returned value
of whatever is represented will be converted to a @code{long}.
@example
@group
@exdent Example:

object  x;
long    val;

x = gNewWithLong(LongInteger, 55L);
val = gLongValue(x);
@end group
@end example
@sp 1
See also:  @code{NewWithLong::LongInteger, ChangeLongValue::Number}
@end deffn










@pdfsubsubsection {ShortValue}
@deffn {ShortValue} ShortValue::Number
@sp 2
@example
@group
s = gShortValue(i);

object  i;
short   s;
@end group
@end example
This method is used to obtain the @code{short} value associated with an
instance of a subclass of the @code{Number} class.  Note that this is
one of the few generics which doesn't return a Dynace object.  It
returns a @code{short}.  If the instance does not represent an instance
of the @code{ShortInteger} class, the returned value of whatever is
represented will be converted to a @code{short}.
@example
@group
@exdent Example:

object  x;
short   s;

x = gNewWithInt(ShortInteger, 55);
s = gShortValue(x);
@end group
@end example
@sp 1
See also:  @code{NewWithInt::ShortInteger, ChangeShortValue::Number}
@end deffn












@pdfsubsubsection {StringRepValue}
@deffn {StringRepValue} StringRepValue::Number
@sp 2
@example
@group
s = gStringRepValue(i);

object  i;
object  s;
@end group
@end example
This method is used to generate an instance of the @code{String} class
which represents the value associated with @code{i}.  This is often
used to print or display the value.  It is also used by
@code{PrintValue::Object} and indirectly by @code{Print::Object}
(two methods useful during the debugging phase of a project)
in order to directly print an object's value.
@example
@group
@exdent Example:

object  x;
object  s;

x = gNewWithInt(ShortInteger, 55);
s = gStringRepValue(x);      /*  s represents "55"   */
@end group
@end example
@sp 1
See also:  @code{PrintValue::Object, Print::Object, FormatNumber::Number}
@end deffn










@pdfsubsubsection {UnsignedShortValue}
@deffn {UnsignedShortValue} UnsignedShortValue::Number
@sp 2
@example
@group
s = gUnsignedShortValue(i);

object  i;
unsigned short  s;
@end group
@end example
This method is used to obtain the @code{unsigned short} value associated
with an instance of a subclass of the @code{Number} class.  Note that
this is one of the few generics which doesn't return a Dynace object.
It returns an @code{unsigned short}.  If the instance does not represent
an instance of the @code{UnsignedShortInteger} class, the returned value
of whatever is represented will be converted to an @code{unsigned short}.
@example
@group
@exdent Example:

object  x;
unsigned short  s;

x = gNewWithUnsigned(UnsignedShortInteger, 55);
s = gUnsignedShortValue(x);
@end group
@end example
@sp 1
See also:  @code{NewWithUnsigned::UnsignedShortInteger,}
@hfil @break @hglue .64in      @code{ChangeUShortValue::Number}
@end deffn










